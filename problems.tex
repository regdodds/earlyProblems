\documentclass[a4paper,12pt]{scrreport}  % Report
%\documentclass[a4paper,12pt]{scrartcl}
\usepackage[left= 72pt, right= 72pt, top= 72pt, bottom= 72pt]{geometry}
% \RequirePackage[2020-02-02]{latexrelease}  % Fixes catoptions endgroup problem NOW
% \usepackage{xkeyval}

\usepackage{amsmath}
\usepackage{amsxtra}

\usepackage{latexsym,
            wrapfig,%psfrag,
            amssymb}
\usepackage{bm}               % for emboldening greek letters in maths $\bm\alpha$
                              %                                   also $\bm{\alpha\beta}$ 
\usepackage{calrsfs}          
                              
\usepackage{mathrsfs}         % Change the style of cal to nicer script
\renewcommand\cal[1]{\mathcal{#1}}
\newcommand{\Cal}{\mathscr}   % Used for set names

\usepackage{graphicx}
\usepackage{color}
\usepackage[table]{xcolor}

\usepackage{longtable}
\usepackage{threeparttable}
% \usepackage{portland}       % to \portrait and \landscape % defunct
\usepackage{makeidx}          % to make the index: \makeindex and \printindex
\usepackage[square]{natbib}   % to get parentheses (...) use [round]
%\usepackage[round]{natbib}   %    Do not simply leave out [square] or [round]
\usepackage{hhline}           % for messing around with parts of an hline
\usepackage{verbatim}         % for comment environment
\usepackage{palatino,lettrine}% dropping capital 
%------------------------------------------------------------------------------
%    add a watermark
%------------------------------>\usepackage[printwatermark]{xwatermark} % requires xcolor and graphicx
%    watermark                 >\newwatermark[allpages,color=gray!20,angle=45,scale=3,xpos=0,ypos=0]{Draft}
%\newwatermark[allpages,textalign=center,fontfamily=pbk,angle=55,scale=2.75,xpos=-1cm,ypos=1cm]{Sample Watermark}
%------------------------------------------------------------------------------

%  \Huge \huge \LARGE \Large \large \normalsize \small \footnotesize \scriptsize \tiny         
\usepackage{listings}
\lstset{language=python, numbers=left, numberstyle=\tiny, numbersep=5pt,
		  basicstyle=\small\sffamily\bfseries, upquote=false, columns=fixed,			% My defaults
%		  basicstyle=\normalsize\sffamily\bfseries,		% OR \small\sffamily,		OR \footnotesize\sffamily,
%																		% OR \scriptsize\sffamily,	OR basicstyle=\tiny\sffamily,
        upquote=false, columns=fullflexible, %columns=fixed,
        backgroundcolor=\color{yellow!20},
        keywordstyle=\bfseries\color{blue},
%		  mathescape=true,		% This is my default, $ in a listing must be escaped
		  mathescape=false,		% This ignores $ in a listing
%		  escapechar={|},			% OR  % escapeinside=||,
        morekeywords={yield,begin,end,function,procedure},
		  literate = {-}{-}1, % Force a short hyphen for the '-' character 2025-10-31
        tabsize=3}
\lstdefinelanguage{Prolog}{
	keywords=[1]{},
	keywordstyle=\color{blue},
}
%       Abbreviations for various lstinline commands
\newcommand\il[1]{% 
\lstinline[basicstyle=\normalsize\sffamily,language={}]!#1!}
\newcommand\ilN[1]{%    ?
\lstinline[mathescape=False,basicstyle=\normalsize\sffamily,language={}]!#1!}
\newcommand\ilb[1]{%
\lstinline[basicstyle=\normalsize\sffamily\bfseries,language={}]!#1!}
\newcommand\ilf[1]{%
\lstinline[basicstyle=\footnotesize\sffamily,language={}]!#1!}
\newcommand\ils[1]{%
\lstinline[columns=fullflexible,basicstyle=\small\sffamily\bfseries,language={}]!#1!}
\newcommand\ilt[1]{%
\lstinline[basicstyle=\tiny\sffamily,language={}]!#1!}

\usepackage{longtable}     % for tables that stretch over more than one page
\usepackage{booktabs}      % for nice tables with \toprule, \midrule and \bottomrule
% Use L{4em}, C{50pt}, or R{2cm}  instead of p{10ex}.  Requires array package.
\usepackage{array}         % for newcolumntypes
\newcolumntype{L}[1]{>{\raggedright\let\newline\\\arraybackslash\hspace{0pt}}m{#1}}
\newcolumntype{C}[1]{>{\centering\let\newline\\\arraybackslash\hspace{0pt}}m{#1}}
\newcolumntype{R}[1]{>{\raggedleft\let\newline\\\arraybackslash\hspace{0pt}}m{#1}}

\usepackage{textcomp}      % for \texttrademark
\def\trademark{{\hbox{\tiny TM}}}
\usepackage{epic}          % for dotted numbering of lists
%\usepackage{graphpap}

%\usepackage{layout}
%   ps \usepackage{pst-func}
%   ps \usepackage{pstricks-add}
\usepackage[pointedenum]{paralist}  
                           % for compactenum which enables 1.1.1 numbering
                           % by providing the compactenum environment
% \usepackage{pstcol}
%\usepackage{mathptmx,bm}             % make mathbf work
%$\alpha\beta\gamma\delta\epsilon$\\
%$\bm{\alpha\beta\gamma\delta\epsilon}$\\
%$\bm\alpha\beta\gamma\delta\epsilon$\\



%\usepackage{layout,layouts}
\usepackage{fancyhdr}

% page typesetting
\pagestyle{fancy}
%\addtolength{\headwidth}{\marginparsep}
%\addtolength{\headwidth}{\marginparwidth}
%%%
%%%
%%%   Only for class book and report
%%%
%%%
%%%\renewcommand{\chaptermark}[1]%
%%%	{\markboth{\thechapter\ #1}{}}
%%%\renewcommand{\sectionmark}[1]%
%%%%	{\markright{\thesection\ #1}}
%%%	{\markright{#1}}

\lhead[\fancyplain{}{\bfseries\thepage}]%
	{\fancyplain{}{\bfseries\rightmark}}
\rhead[\fancyplain{}{\bfseries\leftmark}]%
	{\fancyplain{}{\bfseries\thepage}}
\cfoot{}

\newcommand\Java{\texttt{Java}} % \java 
\newcommand\java{\texttt{Java}} % \java 
\newcommand\python{\texttt{python}} % \java 
\newcommand\class{\texttt{class}} % \class

\newenvironment{Description}
{\begin{list}{}{%
   \setlength\labelwidth{0pt}%
   \setlength\parsep{2pt}%
  \setlength\itemindent{-0.5\leftmargin}%
   \let\makelabel\Descriptionlabel}}%
{\end{list}}% end of definition of Description list environment
\newcommand*\Descriptionlabel[1]{\textbf{#1}\hfil}

\newcommand{\E}{\mbox{E}}
% \renewcommand{\cite}{\citep}

\newcommand{\hash}{\mbox{hash{}}}

\hyphenation{block-chain}
\hyphenation{block-chains}
\hyphenation{cryp-to-gra-phy}
\newcommand\transposed{^{\!\top\!}}
\usepackage{url}

\usepackage[utf8]{inputenc}
\DeclareGraphicsRule{.gif}{.png}{png}{%
  \noexpand\epstopdfcall{convert #1 \noexpand\OutputFile}%
}

\usepackage[font=itshape]{quoting}
\usepackage{fontspec}
\setmainfont{TeX Gyre Termes} % or another font with italics
\setmonofont{DejaVu Sans Mono}[
  Scale=MatchLowercase,
  BoldFont = *-Bold,
  ItalicFont = *-Oblique,
  BoldItalicFont = *-BoldOblique
]
\usepackage{datetime2}
\usepackage{xspace}
\newcommand\longdash[0]{\rule[.5ex]{1.41421em}{0.5pt}\xspace}
%		\newcommand\Ldots[0]{\ldots\xspace}
%     \AppendGraphicsExtensions{.gif}
%     \DeclareGraphicsExtensions{.png,.pdf,.jpg,.jpeg,.gif}
%
%-------------------------------  \newcommand\colorblue[1]{\textcolor{blue}{#1}}
\makeatletter   %  for dots in Contents
\renewcommand\dotfill{\leavevmode\cleaders\hb@xt@.9999em{\hss.\hss}\hfill\kern\z@}
\newcommand\filldots{\leavevmode\cleaders\hb@xt@.4999em{\hss.\hss}\hfill\kern\z@}
\newcommand\tab{\hspace*{2em}}
% Define the allign environment
% -------------------\makeatletter-------------------------------------------->
\newenvironment{allign}
  {\vspace{-\topsep}\start@align\@ne\st@rredfalse\m@ne}
  {\endalign\vspace{-\topsep}}
\newenvironment{allign*}
  {\vspace{-\topsep}\start@align\@ne\st@rredtrue\m@ne}
  {\endalign\vspace{-\topsep}}
%------------------------------------------------------------------------------
\newcommand\rdtitle[1]{%\thispagestyle{empty}% Not needed
\title{\textcolor{blue}{#1}}\author{Reg Dodds\\with the assistance of a robot}\date{\today}}
\newcommand\titleblue[1]{%\thispagestyle{empty}% Not needed
\title{\textcolor{blue}{#1}}\author{Reg Dodds\\with the assistance of a robot}\date{\today}}
\newcommand\blueblue[1]{%\thispagestyle{empty}% Not needed
\title{\textcolor{blue}{#1}}\author{Reg Dodds\\with the assistance of a robot}\date{\today}}
\newcommand\titleBlue[3]{%\thispagestyle{empty}% Not needed
\title{\textcolor{blue}{#1}}\author{#2}\date{#3}}

%		Don't forget a \maketitle

%<--------------------\makeatother---------------------------------------------
\makeatother
\usepackage{enumitem}
\usepackage{relsize}        % for \mathlarger
\usepackage{units}          % for \nicefrac
\usepackage{xfrac}          % for \xfrac and \sfrac

\usepackage{tikz}           % You need to add your own libraries
\usetikzlibrary {arrows.meta,bending,positioning,calc,quantikz2}
% calc                                           calc
%    is for ($(A.east) + (0, 10pt)$) fine positioning
\newenvironment{rdindent}
{\par\leftskip40pt\relax\rightskip40pt\relax}
{\par\leftskip0pt\relax\rightskip0pt\relax}
\newcommand\m{\text{m}}
\newcommand\s{\text{s}}
\newcommand\kg{\text{kg}}

\graphicspath{{figures/}}

\newcommand\sigmoid{\text{sigmoid}}
\newcommand\diag{\text{diag}}
\usepackage[skip=.41\baselineskip]{parskip}
%           \usepackage[skip=2.41\baselineskip]{parskip}
\usepackage{problems}
\begin{document}

\titleBlue{Some problems for CS students to solve}
{Reg Dodds unassisted by robots\\
  Department of Computer Science\\
  University of the Western Cape\\
  \vspace{5pt}
  {\small rdodds@uwc.ac.za}}{2012-12-27---\today}
\begin{comment}
\documentclassa4wide,11pt]{report}
%\usepackage{latexsym,graphicx,wrapfig,psfrag,amssymb}
\usepackage{vaucanson-g,amssymb,amsmath,graphicx,wrapfig,psfrag,letterspace,verbatim,rotating}
\usepackage[round]{natbib}
\usepackage{listings}
\lstset{language=python, numbers=left, numberstyle=\tiny, numbersep=5pt,
%normal   basicstyle=\small\ttfamily, upquote=false, columns=fixed,
        basicstyle=\small\ttfamily, upquote=false, columns=fullflexible, %columns=fixed,
        backgroundcolor=\color{yellow!20},
        keywordstyle=\bfseries\color{blue},
        % escapechar={|},
        escapeinside=||,
        morekeywords={yield,begin,end,function,procedure}}
\input{color.sty}
%\usepackage{pstricks-add}
\usepackage{problems}
%\newcommand{\problem}{\noindent\Large\textbf{Problem}\\*[+6pt]}
%\newcommand{\startsolution{}{\noindent\Large\textbf{Solution}\\*[+6pt]}
\newcommand{\verblst}{\lstinline}
\begin{document}
%\twocolumn
\author{
  Reg Dodds\\
  Department of Computer Science\\
  University of Stellenbosch\\
  \vspace{5pt}
  {\small rdodds@uwc.ac.za}}
\title{Some problems for CS students to solve}
\end{comment}
\maketitle
\renewcommand{\abstractname}{\Large Summary}
\begin{abstract}
This is a set of simple problems that students should be able to solve 
by programming them proficiently in some language.  Most of the solutions 
presented here are presented in a Python-like style or in executable Python
rather than in C-like, or which is similar, Java-like pseudo code.

We have chosen Python because it is so easy to learn.  A pleasant 
side effect is that Python forces users into the tidy habit of indenting 
their code with a rigor not laid down by other languages thus ensuring
clear comprehension of the block structure of their code, uncluttered with 
braces, semicolons, or distracting block delimiters.

% Dit was net
% clear comprehension of the block structure of their code.

Many more solutions than problems are presented because although we 
attempt to evolve an efficient solution to each problem, there is 
generally more than one way of programming it.  Some of the less efficient 
solutions present alternate viewpoints and add to our stock of programming 
tools.  So we often tackle the same problem with solutions that differ 
in outlook and character.  This is not limited to presenting many 
solutions both recursively and iteratively but also to presenting 
dynamic programming or divide-and-conquer solutions.  Sometimes an 
alternate algorithm for the same problem illustrates a fundamentally 
different approach such as deriving the algorithm directly from 
an induction hypothesis, and improving the solution by refining the 
hypothesis.
\end{abstract}
\renewcommand{\abstractname}{\LARGE Preface}
\begin{abstract}
These problems have been lying around in our cupboards in a less accessible 
form.  

James Connan used some of them with a Pascal-like pseudo code that 
does not go down well with Python programmers who are used to a much 
more elegant type of pseudo code that almost looks like the pseudo code 
one finds in modern books on algorithms such as the book, now known as CLRS, 
by~\cite{CLRS2022} that are not based on some specific language and 
that~\cite{CLR1990} have been using since their first edition.  This style of 
layout, that indents the entire block including the final \ilb{END}, 
was already suggested by Marvin Zelkowitz in his version of PL/I called 
PLUM in~\cite{zelkowitz1976}. CLRS, and Python, improved on this by altogether
omitting the end statement.
\end{abstract}

\chapter{Digits and numbers}
The problems in this chapter are based mainly on the properties of numbers.
We consider the digits of a number, the factors of a number, be they 
prime or not, the primacy of a number, and other related problems. 
\startproblem{Count the number of digits in integer $n$}
Calculate the number of decimal digits of the integer $n$. 
\problemend

\startsolution{Counting the digits of the integer $n$}
The digits of $n$ are manipulated by integer division and multiplication 
of $n$ by $10$.   We indicate integer division by `//'.\footnote{Python 
3 uses the binary operator `//', while most other languages, even earlier 
versions of Python, use the binary operator `/' embedded between two integers
to give an integer result.  When one of the operands of `/' is a floating point 
number then the result also becomes floating point and requires a function
such as \texttt{int()} to convert it to an integer.} So, \lstinline!n = n // 10! 
\textit{removes} the last digit of $n$.\footnote{Experienced programmers 
conventionally write \lstinline!n = n // 10! as \lstinline!n //= 10!.} 
To retrieve this removed digit necessitates some manipulation.  First, 
consider the number \lstinline!(n // 10) * 10!.  It differs from $n$ in having 
its final digit replaced by $0$;  the other digits are the same.  So, to get 
the final digit, all we need to do is to deduct it from $n$ as follows
\begin{lstlisting}[keywordstyle=\texttt]
digit = n - (n // 10) * 10
\end{lstlisting}
Another obvious way of calculating the last digit is to use the \textit{modulo} 
operator, which is written as \texttt{\%} in Python as follows:
\begin{lstlisting}[keywordstyle=\texttt]
digit = n % 10
\end{lstlisting}
But we were not asked to get the last digit,  we were asked to count 
all the digits of $n$.  This is done by retrieving the last digit 
repeatedly and counting these iterations as follows.  We initialize the 
number of digits $m$ to zero and at each step replace $n$ by its value 
with its last digit removed.
\begin{lstlisting}[keywordstyle=\texttt]
m = 0
while n > 0:
  n = n // 10
  m = m + 1
\end{lstlisting}
However, this will usually be programmed by using the `operate-and-becomes' 
notation and embedding the code in a function.
\begin{lstlisting}[keywordstyle=\texttt]
def noDigits(n):
  m = 0
  while n > 0:
    n //= 10
    m += 1
  return m

m = noDigits(n)
\end{lstlisting}
The variable $n$ is unaffected by the invocation \lstinline!m = noDigits(n)!.
This is not the last word.  Python has some useful built-in operators 
that should be used when programming this problem.  The \lstinline!str()! 
function can be used to turn the integer into a string, and the \lstinline!len()! 
function counts the number of characters in the string. The call below 
gives the same result.
\begin{lstlisting}[keywordstyle=\texttt]
m = len(str(n))
\end{lstlisting}
\begin{exercise}
Display the digits of the integer in reverse.
\end{exercise}
\solutionend

\startproblem{Given integers $n$ and $m$ calculate $n^m$}
%The definition of $n$ to some positive integer power $m$ is
When $m$ is positive $n^m$ or $n$ to the power $m$ is defined as 
\[ n^m = n \times n \times n \times \ldots\ m \mbox{ \textup{times}}.\]
More generally, the definition of $n$ to any integer power $m$ is 
\[ n^m = \left\{ \begin{array}{ll}
                       n \times n \times n \times \ldots\ m \mbox{ \textup{times}}        & \mbox{ \textup{if} } m > 0,\\
                       1                                                          & \mbox{ \textup{if} } m = 0, \\
                       1/ (n \times n \times n \times \ldots\ |m| \mbox{ \textup{times}}) & \mbox{ \textup{if} } m < 0.
                 \end{array}\right.\]

\problemend

\startsolution{Number raised to integer power}
For positive values the power is quite easy to calculate. Simply start with a \texttt{product} that is $1$ and 
replace it \texttt{m} times by $\texttt{product} \times \texttt{n}$, i.e., 
\begin{lstlisting}[keywordstyle=\texttt] 
product = 1
for count in range(m):
  product = product * n
\end{lstlisting} 
This seems to be the last word, but it is rather inefficient.  We will discuss more
efficient ways of doing later in Problem~\ref{fastexponentiation}.  Now we will use
the function below to do integer exponentiation.  It takes $m$ multiplications.
\begin{lstlisting}[keywordstyle=\texttt] 
def power(n, m):
  product = 1
  for count in range(m):
    product = product * n
  return product
\end{lstlisting} 
When the power is negative we proceed in the same way for $|m|$ multiplications
but invert the product. We use a boolean variable called \texttt{negativePower} that becomes 
\texttt{True} when the power is negative to keep track of what action to take. 
\begin{lstlisting}[keywordstyle=\texttt] 
def power(n, m):
  product = 1
  negativePower = False
  if m < 0:
    negativePower = True
    m = -m
  for count in range(m):
      product = product * n
  if negativePower:
    product = 1/product
  return product
\end{lstlisting} 
\solutionend
\startsolution{Number raised to integer power}
The power $n^m$ can be defined recursively as 
\[n^m = \left\{ \begin{array}{ll}
                       n \times n^{m-1}   & m > 1\\
                       1                  & m = 0\\
                       1/n^{m+1}          & m < 1\\
                 \end{array}\right.\]

\solutionend

\startproblem{Check if $n$ is an Armstrong number}
Design an algorithm to check if a given number $n$ is an Armstrong number.
\problemend
Suppose the number $n$ is written in decimal form as
\[n = d_1^{m-1}10^{m-1} + d_2^{m-2}10^{m-2} + \ldots + d^2_{m-2}10^2 + d_{m-1}10 + d_m\]
where each $d_i \in \{0,1,2,3,4,5,6,7,8,9\}$ for $i \in [1..m]$ and 
$m$ is the number of decimal digits of $n$.

\noindent The integer $n$ is an Armstrong number if the sum of each digit of 
$n$ raised to the power of the number of digits in the number, is 
equal to the original number, then that number is said to be an Armstrong 
number,  that is, $n$ is an Armstrong number if the following equality holds
\[n = d_1^m + d_2^m + \ldots + d_{m-2}^m + d_{m-1}^m + d_m^m\]


\startsolution{Algorithm for Armstrong number}
Given the integer $n$, the program must split it into digits, count 
them, and then sum their powers and check if the sum equals $n$.

\noindent The digits of $n$ are peeled of it from the rear.  The least significant 
digit of $n$ is found by integer division and multiplication.  Integer
division gives an integer result by dropping the fractional part of the 
division.  Multiplying this result by the divisor produces the original 
number with a zero in the last digit.  Deducting this zero-ended number
from $n$ yields the least significant digit---the normally most
right-hand digit of $n$.
\begin{lstlisting}[keywordstyle=\texttt]
digit = n - (n // 10) * 10
\end{lstlisting}
In order to proceed to the next-to-last digit the last digit needs to 
be removed.  Integer division by $10$ gets rid of the last digit.
We adapt our code slightly so that the final digit can be repeatedly 
removed until nothing---zero---is left.  The following code must
be iterated to get each digit.
\begin{lstlisting}[keywordstyle=\texttt]
digit = n - (n // 10) * 10
n = n // 10
\end{lstlisting}
By introducing a new variable $ndiv10$ we can avoid doing the division 
twice.  The \lstinline!while! statement runs until $n$ becomes zero.
\begin{lstlisting}[keywordstyle=\texttt]
while n > 0:
  ndiv = n // 10
  digit = n - ndiv10 * 10
  n = ndiv10
\end{lstlisting}
After running the while only the leftmost digit will be available, 
so we need to do the summation for the Armstrong number inside the
loop.
\begin{lstlisting}[keywordstyle=\texttt]
Armstrong = 0
while n > 0:
  ndiv = n // 10
  digit = n - ndiv10 * 10
  Armstrong += digit ** m
  n = ndiv10
\end{lstlisting}
This will not work because we have know the value of $m$ before the 
loop starts. A similar loop is necessary to determine $m$ but we must 
be careful not to destroy $n$ in the process.
\begin{lstlisting}[keywordstyle=\texttt]
m = 0
copyOfn = n
while n > 0:
  ndiv = n // 10
  m += 1
n = copyOfn
# followed by Armstrong code
\end{lstlisting}

% James untouched
\begin{lstlisting}[keywordstyle=\texttt]
digits : array[1..100] of number

get num from user
set cnt = 0
while num <> 0 do
  begin block
     set cnt = cnt + 1
     set digits[cnt] = remainder of num/10
     set num = integer part of num/10
  end block
set total = 0
for i ranges from 1 to cnt do
  set total = total + digits[i]\^cnt
if num = total then display num is an Armstrong number
\end{lstlisting}
\solutionend

\startproblem{Display the first 15 Armstrong numbers}
\problemend

\begin{lstlisting}[keywordstyle=\texttt]
set number = 0
set num = 1
repeat
  set cnt = 0
  while num <> 0 do
    begin block
      set cnt = cnt + 1
      set digits[cnt] = remainder of num/10
      set num = integer part of num/10
    end block
  set total = 0
  for i ranges from 1 to cnt do
    set total = total + digits[i]^cnt
  if num = total then 
    begin block
      display num 
      set number = number + 1
    end block
  set num = num + 1
until number = 15
\end{lstlisting}

\startproblem{Find all the $n < 10000$ with nondecreasing digits}
Examples of such numbers are $11, 12, \dots, 19, 22, 23, \dots, 29, 33, 34, \ldots, 39$. 

\problemend

\startproblem{Find all the $n < 1000$ with nondecreasing digits that also 
have squares with nondecreasing digits}
Example of such numbers are $12^2 = 144, 13^2=169, 15^2=225, 16^2=256, 37^2 = 1369$. 
\problemend

\startproblem{Convert an integer in base $10$ to another base} 
Given a number in base $10$, convert it to a base supplied by the user. 
 The user supplied base can be anything between $2$ and $16$.
\problemend


\startsolution{Converting between bases}
If there was a limit on the size of the base 10 number, then a collection of if statements would have sufficed.  That is:

\begin{lstlisting}[keywordstyle=\texttt]]
get Num from user
get Base from user

if Num > Base^n then
    begin block
        display integer part of Num/(Base^n)
        set Num = remainder of Num/(Base^n)
    end block
    .
    .
    .
if Num > Base^0 then
    begin block
        display integer part of Num/(Base^0)
        set Num = remainder of Num/(Base^0)
    end block


where Base^n is Base to the power of n
\end{lstlisting}
\solutionend

\startsolution{2}
This can easily be rewritten as 
\begin{lstlisting}[keywordstyle=\texttt]]
get Num from user
get Base from user


for n ranges from x to 0 do
    begin block
        if Num > Base^n then
            begin block
                display integer part of Num/(Base^n)
                set Num = remainder of Num/(Base^n)
            end block
    end block
\end{lstlisting}
\solutionend

However, we are still restricted by the size of $x$ and $2^10 (1024)$ 
is much smaller than $9^10 (3486784401)$

\startsolution{3}

We know that we can convert a number from base ten to any base by repeatedly dividing by the base.  So we have


\begin{lstlisting}[keywordstyle=\texttt]]
get Num from user
get Base from user

repeat
    display remainder Num/Base
    set Num = integer part of Num/Base
until Num = 0
\end{lstlisting}
\solutionend


\startsolution{4}

As a while loop

\begin{lstlisting}[keywordstyle=\texttt]]
get Num from user
get Base from user

while (Num <> 0) do
    begin block
        display remainder Num/Base
        set Num = integer part of Num/Base
    end block
Implement a binary search algorithm that displays 
the index of the desired value and the number of 
lookups made.Implement a bubble sort algorithm.Write a program that allows the user 
to play blackjack against the computer.
\end{lstlisting}
\solutionend

Use a array of unique random numbers to 
"shuffle" the cards.d

\startproblem{Cash register}

When a cash purchase is made, the cashier normally has to give the customer change.  The customer does not want a bulky wallet, so it is up to the cashier to give the customer the smallest number of notes and coins.  Write an algorithm to assist the cashier in determining which notes and coins to give the customer.
\problemend

Answer

What do we need?
The total of the purchase made.
The amount tendered.

What do we know?
The following notes and coins are available to us:
R200, R100, R50, R20, R10, R5, R2, R1, 50c, 20c, 10c, 5c, 2c, 1c

What do we want to do?
We want to minimize the number of coins and notes used, so we want to always use the biggest (in value) notes and coins possible.  So we start with the biggest denomination and work our way down, each time calculating how many of that denomination to return.

\startsolution{1}

\begin{lstlisting}[keywordstyle=\texttt]]
get purchase Total from user
get Amount tendered from user
set Change = Amount - Total

if Change >= R200 then
    begin block
        display R200 x integer part of Change/R200
        set Change = remainder of Change/R200
    end block
if Change >= R100 then
    begin block
        display R100 x integer part of Change/R100
        set Change = remainder of Change/R100
    end block
if Change >= R50 then
    begin block
        display R50 x integer part of Change/R50
        set Change = remainder of Change/R50
    end block
if Change >= R20 then
    begin block
        display R20 x integer part of Change/R20
        set Change = remainder of Change/R20
    end block
if Change >= R10 then
    begin block
        display R10 x integer part of Change/R10
        set Change = remainder of Change/R10
    end block
if Change >= R5 then
    begin block
        display R5 x integer part of Change/R5
        set Change = remainder of Change/R5
    end block
if Change >= R2 then
    begin block
        display R2 x integer part of Change/R2
        set Change = remainder of Change/R2
    end block
if Change >= R1 then
    begin block
        display R1 x integer part of Change/R1
        set Change = remainder of Change/R1
    end block
if Change >= 50c then
    begin block
        display 50c x integer part of Change/50c
        set Change = remainder of Change/50c
    end block
if Change >= 20c then
    begin block
        display 20c x integer part of Change/20c
        set Change = remainder of Change/20c
    end block
if Change >= 10c then
    begin block
        display 10c x integer part of Change/10c
        set Change = remainder of Change/10c
    end block
if Change >= 5c then
    begin block
        display 5c x integer part of Change/5c
        set Change = remainder of Change/5c
    end block
if Change >= 2c then
    begin block
        display 2c x integer part of Change/2c
        set Change = remainder of Change/2c
    end block
if Change >= 1c then
    begin block
        display 1c x integer part of Change/1c
        set Change = remainder of Change/1c
    end block
\end{lstlisting}
\solutionend



\startsolution{2}

The algorithm above solves the problem, but is not very elegant.  It would be nice if we could find a way of using a loop rather than writing so many if statements.  We notice that the pattern 5xx, 2xx, 1xx is repeated.  In fact, if we had a R500 note, we would have had the same pattern 5xx, 2xx, 1xx repeated 5 times, decreasing by a factor 10 every time.  So we will use this to to implement a more elegant solution.  In order to make it easier for reader to follow, it will be assumed that all amounts are in cents instead of Rands and cents.  (Converting all amounts to cents can easily be achieved multiplying by 100.)

\begin{lstlisting}[keywordstyle=\texttt]]
get purchase Total from user
get Amount tendered from user
set Change = Amount - Total

set One = 50000
set Two = 20000
set Three = 10000

repeat
   if (Change >= One) and (One < 50000) then
       begin block
           if One >= 500 then display R(One/100) x integer part of Change/One
           else display (One)c x integer part of Change/One
           set Change = remainder of Change/One
           set One = One/10
       end block
   if (Change >= Two) then
       begin block
           if Two >= 200 then display R(Two/100) x integer part of Change/Two
           else display (Two)c x integer part of Change/Two
           set Change = remainder of Change/Two
           set Two = Two/10
       end block
   if (Change >= Three) then
       begin block
           if Three >= 100 then display R(Three/100) x integer part of Change/Three
           else display (Three)c x integer part of Change/Three
           set Change = remainder of Change/Three
           set Three = Three/10
       end block
until Change = 0
\end{lstlisting}
\solutionend



\startsolution{3}

The algorithm can be rewritten using a while loop.

\begin{lstlisting}[keywordstyle=\texttt]]
get purchase Total from user
get Amount tendered from user
set Change = Amount - Total

set One = 50000
set Two = 20000
set Three = 10000

while Change <> 0) do
   begin block
       if (Change >= One) and (One < 50000) then
           begin block
               if One >= 500 then display R(One/100) x integer part of Change/One
               else display (One)c x integer part of Change/One
               set Change = remainder of Change/One
               set One = One/10
           end block
       if (Change >= Two) then
           begin block
               if Two >= 200 then display R(Two/100) x integer part of Change/Two
               else display (Two)c x integer part of Change/Two
               set Change = remainder of Change/Two
               set Two = Two/10
           end block
       if (Change >= Three) then
           begin block
               if Three >= 100 then display R(Three/100) x integer part of Change/Three
               else display (Three)c x integer part of Change/Three
               set Change = remainder of Change/Three
               set Three = Three/10
           end block
    end block
\end{lstlisting}
\solutionend



\startsolution{4}

The algorithm in Solutions 2 and 3 still uses multiple if statements.  We can reduce this by recognizing another pattern.

R200.00 R100.00 R50.00
R20.00  R10.00  R5.00
R2.00   R1.00   R0.50
R0.20   R0.10   R0.05
R0.02   R0.01

We can obtain the next denomination by halving the current one.  Every third time we multiply by 2/5.  We continue to use only cents and the algorithm now becomes:

\begin{lstlisting}[keywordstyle=\texttt]]
get purchase Total from user
get Amount tendered from user
set Change = Amount - Total

set Denomination = 200000
set Cnt = 1

while (Change <> 0) do
   begin block
       if (Change >= Denomination) then
           begin block
               if Denomination >= 100 then display R(Denomination/100) x integer part of Change/Denomination
               else display (Denomination)c x integer part of Change/Denomination
               set Change = remainder of Change/Denomination
           end block
       if Cnt is divisible by 3 then set Denomination = 2*Denomination/5
       else set Denomination = Denomination/2
       set Cnt = Cnt + 1
    end block
\end{lstlisting}
\solutionend

\startsolution{5}

We can also use our knowledge of arrays to solve the problem.  We use a two dimensional array to store the number of each denomination needed.  First we initialize the array with the denominations and we zero the number for each.


Money is an array[1..2][1..14] of number

\begin{lstlisting}[keywordstyle=\texttt]]
get purchase Total from user
get Amount tendered from user
set Change = Amount - Total

set Denomination = 20000

for Index ranges from 1 to 14 do
    begin block
        set Money[1][Index] = Denomination
        set Money[2][Index] = 0
        if Index is divisible by 3 then set Denomination = 2*Denomination/5
        else set Denomination = Denomination/2
    end

set Cnt = 1

while (Change <> 0) do
    begin block
        if Change >= Money[1][Cnt] then
            begin block
                set Money[2][Cnt] = integer part of Change/Money[1][Cnt]
                set Change = remainder of Change/Money[1][Cnt]
            end block
        set Cnt = Cnt + 1
    end block

set Cnt = 1

while (Money[1][Cnt] >= 100) do
    begin block
        if Money[2][Cnt] > 0 then display R(Money[1][Cnt]/100) x Money[2][Cnt]
        set Cnt = Cnt + 1
    end block

repeat
    if Money[2][Cnt] > 0 then display (Money[1][Cnt])c x Money[2][Cnt]
    set Cnt = Cnt + 1
until Cnt > 14
\end{lstlisting}
\solutionend

\startproblem{Determine if a number $N$ is prime}
Given a number (N), write an algorithm to determine if the number is prime or not.
\problemend


What is a prime number?

It is a number that is divisible only by itself and $1$.

How would we check to see if a number is prime or not?
If a number is divisible by any number other than 1 or itself, then that number is not prime.

\startsolution{Algorithm for primes by trial division}

In plain English

Get the number to check

Assume the number is prime

Test divisibility of the number by all potential factors between 1 and the number itself

If the number is divisible by any of these numbers, then it is not prime, otherwise it is prime

In pseudo code

\begin{lstlisting}[keywordstyle=\texttt]]
get Num from user
get IsPrime = True
for PFactor ranges from 2 to Num-1 do
  begin block
     if Num divisible by PFactor then set IsPrime = False
  end block
if IsPrime = True then display Num is prime
else display Num is not prime
\end{lstlisting}
\solutionend


\startsolution{Slightly improved prime checker}

If we look at the factors of non-prime numbers, we notice that they 
are symmetrical around the square root of the number.  That is, for 
say $36$, the factors are $1\times36$, $2\times18$, $3\times12$, $4\times9$, 
$6\times6$, $9\times4$, $12\times3$, $18\times2$, and $36\times1$. 
 Therefore we only need to check each potential factor up 
to the square root of the number.  Also, as soon as we find a factor 
for the number, we can stop checking because the number is not prime.

In plain English

Get the number to check

Check all potential factors until the factor is greater than the square 
root of the number or we find a factor that divides into the number 
without a remainder

In pseudo code

\begin{lstlisting}[keywordstyle=\texttt]]
repeat loop

get Num from user
set PFactor = 1
repeat
  set PFactor = PFactor + 1
until (PFactor > square root of (Num)) or (Num divisible by PFactor)
if PFactor <= square root of (Num) then display Num is not prime
else display Num is prime


while loop

get Num from user
set PFactor = 2
while (PFactor <= square root of (Num)) and (Num not divisible by PFactor)
  begin block
     set PFactor = PFactor + 1
  end block
if PFactor <= square root of (Num) then display Num is not prime
else display Num is primeWrite a program that will convert and angle from 
decimals to degrees minutes and seconds.Write a program that will convert a number from 
digits to its English equivalent.  For example 105 becomes
one hundred and five.GCD
\end{lstlisting}
\solutionend

\startproblem{Greatest common divisor---GCD}
Write and algorithm to find the GCD (Greatest Common Divisor) of two 
numbers, $x$ and $y$.
\problemend

\startsolution{1}

What is the GCD?

It is the biggest number that divides into both $x$ and $y$ without 
leaving a remainder.

This means that the biggest possible GCD is at most the smaller of 
the two numbers.


\begin{lstlisting}[keywordstyle=\texttt]]
get x from user
get y from user

if x > y then set Smaller = y
else Smaller = x

for PFactor ranges from 1 to Smaller do
    begin block
        if (x divisible by PFactor) and (y divisible by PFactor) then GCD = PFactor
    end block
display GCD
\end{lstlisting}
\solutionend



\startsolution{GCD version $2$}
We can reduce the execution time of the algorithm by first checking 
if the smaller number is the GCD and then checking all numbers less 
then it and stopping as soon as we find a factor.


\begin{lstlisting}[keywordstyle=\texttt]]
get x from user
get y from user

if x > y then set Smaller = y
else Smaller = x

set GCD = Smaller

While not((x divisible by GCD) and (y divisible by GCD)) do
    set GCD = GCD - 1
display GCD
\end{lstlisting}


The condition for the while statement can also be written as:
While (x not divisible by GCD) or (y not divisible by GCD) do
\solutionend


\startsolution{3}
Two millennia and three centuries ago Euclid presented us with an even 
more elegant solution.  He showed that

\begin{lstlisting}[keywordstyle=\texttt]]
GCD(x, y) = GCD(y, remainder of x/y)  if y > 0
GCD(x, y) = x        if y = 0

In pseudo code this is:

get x from user
get y from user

while (y > 0) do
    begin block
        set Tmp = remainder of x/y
        set x = y
        set y = tmp
    end block
display GCD = x
\end{lstlisting}
\solutionend


\startproblem{GCD of a list of numbers}
Find the GCD of any list of numbers.
\problemend

\startsolution{GCD of a list numbers}
The GCD of three or more numbers can be found by finding the GCD of 
pairs of numbers.  That is:

\begin{lstlisting}[keywordstyle=\texttt]]
GCD (x, y, z) = GCD (x, GCD (y, z)) or GCD (x, y, z) = GCD ( GCD (x, y), z)
\end{lstlisting}
\solutionend


\startsolution{GCD of a list numbers}
If can also be found using the same strategy as used in Solutions 
1 and 2.  Solution 2, for three numbers would then be

\begin{lstlisting}[keywordstyle=\texttt]]
get x from user
get y from user
get z from user

if (x > y) and (z > y) then set Smallest = y
if (x > z) and (y > z) then set Smallest = z
if (y > x) and (z > x) then set Smallest = x


set GCD = Smaller

While not((x divisible by GCD) and (y divisible by GCD) and (z divisible by GCD)) do
    set GCD = GCD - 1
display GCD
\end{lstlisting}
\solutionend

\startproblem{Convert marks into letter grades}
Write a program that will convert a mark as a given as a percentage to grade 
symbol, e.g. 80\% becomes an `A'. 
\problemend



\startproblem{Guess the number}
Write a program that will generate a 
random integer, in the range, starting with $[1..10]$
and allow the user to guess the number.  

After each guess the program tells
the user if the guess was to high or too
low.  After 12 or so unsuccessful
guesses the game is lost. 

Introduce levels so that each time the 
user guesses the number correctly, a new
number is generated in from a larger 
range.  Each time the user fails to guess
the correct number within the allowed 
number of guesses, they get demoted to 
a lower level.
\problemend


\startproblem{Lowest common denominator---LCM}
Find the LCM (Lowest Common Denominator) of two numbers, x and y.

What is the LCM?

It is the smallest number that is divisible by both $x$ and $y$.  
That is, into which both $x$ and $y$ will divide without leaving a 
remainder.

What do we know about the LCM?

The LCM must be greater than or equal to the larger of $x$ and $y$. 
 The LCM must be less than or equal to the product of the $x$ and $y$.
\problemend


\startsolution{LCM 1}

\begin{lstlisting}[keywordstyle=\texttt]]
get x from user
get y from user

if x > y then set Greater = x
else set Greater = y


for PLCM ranges from (x*y) to Greater do
    if (PLCM is divisible by x) and (PLCM is divisible by y) then set LCM = PLCM
display LCM
\end{lstlisting}
\solutionend


\startsolution{2}
Solution 1 will always check all numbers between the larger (of x 
and y) and the product (of x and y).  We can also rewrite the algorithm 
in such a way that it will only check numbers up to and including 
the LCM and then stop.

\begin{lstlisting}[keywordstyle=\texttt]]
get x from user
get y from user

if x > y then set Greater = x
else set Greater = y

set PLCM = Greater
set LCM = 1

repeat
    if (PLCM is divisible by x) and (PLCM is divisible by y) then set LCM = PLCM
    set PLCM = PLCM + 1
until (LCM is divisible by x) and (LCM is divisible by y)
display LCM
\end{lstlisting}

Note: LCM is used to terminate the loop because the value of PLCM 
is changed after assigning it to LCM.  Also, LCM is initialized otherwise 
it has no definite value until the condition for the if statement 
is satisfied.
\solutionend

\startsolution{LCM 3}
We know that if one number is divisible by another number, then the 
first number must be a multiple of the second number.  That is, 100 
is divisible by 10.  100 is a multiple of 10.  10 x 10 = 100.  Only 
multiples of 10 will be divisible by 10 (10, 20, 30, ...).  We therefore 
do not even need to check numbers between 10 and 20, or 20 and 30, 
etc.  Solution 2 can therefore be improved as follows:

\begin{lstlisting}[keywordstyle=\texttt]]
get x from user
get y from user

if x > y then set Greater = x
else set Greater = y

set PLCM = Greater
set LCM = 1

repeat
    if (PLCM is divisible by x) and (PLCM is divisible by y) then set LCM = PLCM
    set PLCM = PLCM + Greater
until (LCM is divisible by x) and (LCM is divisible by y)
display LCM
\end{lstlisting}
\solutionend


\startsolution{LCM 4}

Solution 3 can be rewritten using a while loop.


\begin{lstlisting}[keywordstyle=\texttt]]
get x from user
get y from user

if x > y then set Greater = x
else set Greater = y

set PLCM = Greater

while not ((PLCM is divisible by x) and (PLCM is divisible by y)) do
    set PLCM = PLCM + Greater

display PLCM
\end{lstlisting}

Note:  We are checking the loop termination condition before changing 
PLCM, so we do not need to make a copy of it.  When the loop terminates, 
PLCM will be the LCM.  Also, the not can be moved into the bracket, 
changing the condition from not((PLCM is divisible by x) and (PLCM 
is divisible by y)) to (PLCM is not divisible by x) or (PLCM is not 
divisible by y)
\solutionend


\startproblem{Solve a set of $n$ linear equations}
Write a program that will solve a set of $n$ linear 
equations in $n$ unknowns.  Write a program that allows the user to perform 
matrix operations, such as matrix addition, subtraction,
and multiplication.  Write a program that generates two random arrays 
(reuse your old code) of length $n$ and $m$ respectively.  
Merge the two arrays into one by alternately selecting 
elements from each.Write a program that takes two sorted arrays and 
produces a single sorted array.
\problemend

\startproblem{Merge two sorted lists}
Given arrays sorted n ascending order,  merge them to produce one 
sorted array.
\problemend



\startsolution{Merge two sorted lists}

\begin{lstlisting}[keywordstyle=\texttt]]
arrone : array [1..n] of number
arrtwo : array [1..m] of number
arrthree : array [1..(m+n)] of number

procedure merge
  begin block
    set cnt1 = 1
    set cnt2 = 1
    set cnt3 = 1
    while (cnt1 <= n) and (cnt2 <= m) do
       begin block
         if arrone[cnt1] < arrtwo[cnt2] then
           begin block
             set arrthree[cnt3] = arrone[cnt1]
             set cnt1 = cnt1 + 1
           end block
         else
           begin block
             set arrthree[cnt3] = arrtwo[cnt2]
             set cnt2 = cnt2 + 1
           end block
         set cnt3 = cnt3 + 1
       end block
    if cnt1 > n then
      begin block
        for i ranges from 0 to m-cnt2 do
          set arrthree[cnt3+i] = arrtwo[cnt2+i] 
      end block
    else 
      begin block
        for i ranges from 0 to n-cnt1 do
          set arrthree[cnt3+i] = arrone[cnt1+i]
       end block
  end block
\end{lstlisting}
\solutionend

\startproblem{Find all the prime numbers from $2$ to $N$.}
\problemend

\noindent There are many useful variations of this problem.

\startproblem{Find the first N primes}
\problemend


\startsolution{Find primes}
\noindent What is a prime number?

\noindent It is a number that is divisible only by itself and 1.

\noindent How would we check to see if a number is prime or not?

\noindent If a number is divisible by any number other than 1 or itself, 
then that number is not prime.

\noindent What do we know/have?

\noindent We have already constructed several algorithms to check whether a 
number is prime or not. All we need to know now is use one of these 
algorithms to check all numbers and count each prime until we have N of them.

Algorithm

\noindent In plain English

\noindent Get $N$, the number of primes the user wishes to find
Check all numbers from $2$ up until we have found $N$ primes

\noindent In plain English, refine

\begin{lstlisting}[keywordstyle=\texttt]]
Get N, the number of primes the user wishes to find

Start with 2
Repeat
  Assume the number is prime
  Test divisibility of the number by all possible numbers
  between 1 and the number itself
  If the number is divisible by one of these numbers, then 
    it is not prime,
  Otherwise 
    it is prime, 
    count it
  Until we have found N primes
\end{lstlisting}


In slightly more exact pseudo code

\begin{lstlisting}[keywordstyle=\texttt]]
repeat loop

get N from user
set Count = 0
set Num = 2
repeat
  set PFactor = 1
  repeat
    set PFactor = PFactor + 1
  until (PFactor > square root of (Num)) or (Num divisible by PFactor)
  if (PFactor <= square root of (Num)) then display Num is not prime
  else 
    begin block
      display Num is prime
      set Count = Count + 1
    end block
until Count = N
\end{lstlisting}

Another more refined version

\begin{lstlisting}[keywordstyle=\texttt]]
while loop

get N from user
set Count = 0
set Num = 2
while Count < N do begin block
    set PFactor = 2
    while (PFactor < square root of (Num)) and
          (Num not divisible by PFactor) do
      set PFactor = PFactor + 1

    if (PFactor <= square root of (Num)) then 
      display Num is not prime
    else begin block
        display Num is prime
        set Count = Count + 1 end block end block
\end{lstlisting}
\solutionend


\startproblem{Numeric integration by summing rectangles}
Given a function $y = f(x)$, an interval $[x_i, x_{i+1}]$, 
and a number $n$ of intervals such that $i = 0, 1, \ldots, n-1$.
\problemend

\noindent Find the approximate area under the curve by calculating 
and summing the areas of each rectangle of width $x_i-x_{i+1}$
and height $f(x_{i})$.

\startsolution{Integration}
\noindent We slice the area under the graph into $n$ rectangles and 
calculate the area of each rectangle.  Summing the areas of each rectangle 
will yield an approximation for the total area under the graph.

\noindent If we substitute a value for $x$ into the equation, we get 
the corresponding $y$ value.  This $y = f(x_i)$ value represents the 
height of the rectangle.  Using $x_i$ and $x_{i+1}$ we can calculate 
the width.  Thus we can calculate the area of the rectangle.  If we 
sum the areas of all these rectangles that are in the interval $[a,b]$ 
then we get the approximate area under the graph.  It should be clear
that the sum of these rectangles is less than the area under the curve.

\begin{lstlisting}[keywordstyle=\texttt]]
get f from user
get a from user
get b from user
get n from user

set Area = 0
for Tmp ranges from a to b in steps of (b - a)/n do
  set Area = Area + f(Tmp)*width
display Area
\end{lstlisting}
\solutionend

We can improve this approximation by using the second ordinate for 
the value of the height, and following the same procedure.   This 
approximates the actual area under the curve from above.  The average 
of these two values will provide a much better approximation of the 
integral.  Without repeating the whole process the following idea 
achieves the same result.

\startproblem{Numeric integration by summing trapezia\footnote{If 
you do not know Latin you might be inclined to spell the plural of 
trapezium  as \textit{trapeziums}.  Some people talk of \textit{trapezoids}.}}
\noindent This problem is often posed as finding an approximation 
to the definite integral
\[\int_a^b f(x)\mbox{d}x.\]
\noindent Find the approximate area under the curve by calculating 
and summing the areas under each trapezium bounded by the $x$-axis, by 
the two ordinates $x_i$--$f(x_i)$, $x_{i+1}$--$f(x_{i+1})$,  and 
the line $(x_i, f(x_i))$--$(x_{i+1}, f(x_{i+1}))$.  Figure\ref{fig:trapezium} 
\begin{figure}[h]
\begin{center}
%{%Start of psfrag
%\psfrag{fofx}{$f(x)$}
%\psfrag{xi}{$x_i$}
%\psfrag{xpi}{$x_{i+1}$}
%\psfrag{fxi}{$f(x_i)$}
%\psfrag{fxpi}{$f(x_{i+1})$}
%\include{figures/trapezium.tikz}
%\include{figures/trapezium.tikz}
%\includegraphics[scale=1.0]{figures/trapezium}%}%End of psfrag
\input{figures/trapezium.pdf_t}
% To get the .pdf_t file from XFIG you need to export using the option
% Combined PDF.LaTeX (both parts)
% EDIT the .pdf_t file by adding [scale=1.0] or whatever scale you need
% If you don't add the scale nothing appears in the version of the
% lualatex/pdflatex/... engine I am using.
\end{center}
\caption{A trapezium.}\label{fig:trapezium}
\end{figure}
shows the $i$th trapezium.  The function $f(x)$ is given. and the 
bounds of the area to be approximated are given as the closed interval $[a, b]$.
\problemend


\chapter{Binomial coefficients and Pascal's triangle}
\section{Binomial coefficients}
\label{sec:binomialcoefficients}
Consider the binomial $(x + y)^n$ for $n = 0$. Since $(x + y )^0 = 1$, the terms
in $x$ and $y$ disappear and we write down a single $1$.  Putting $n= 1$
\[(x+y)^1 = 1x + 1y,\]
and we get a $1$ and another $1$.  When $n = 2$ the equation becomes
\[(x+y)^2 = 1x^2 + 2xy + 1y^2\] giving $1$, $2$, and $1$ again.  Next for 
$n = 3$ the equation becomes
\begin{eqnarray*}
(x+y)^3 &=& (x+y)(1x^2 + 2xy + 1y^2)\\
        &=& 1x^3 + 2x^2y + 1xy^2 \\
        &=& \mbox{\hspace{17pt}}+  1x^2y + 2xy^2 + 1y^3\\
        &=& 1x^3 +  3x^2y + 3xy^2 + 1y^3,
\end{eqnarray*}
yielding the coefficients $1$, $3$, $3$, and $1$.  In general we write
\begin{equation}
(x+y)^n = \sum_{r = 0}^n \binom{n}{r}x^ry^{n-r}\label{eqn:binarycoefficient}
\end{equation}
giving the coefficients as the numbers $\binom{n}{r}$.  For now we 
know that $\binom{3}{0} = 1$,   $\binom{3}{1} = 3$,   $\binom{3}{2} = 3$,   
$\binom{3}{3} = 1$.  One of the tasks we set is to calculate these 
coefficients in various ways.  Before doing this we look at a property
that we can derive from Equation~\ref{eqn:binarycoefficient} by setting both 
$x$ and $y$ to $1$,
\begin{eqnarray*}
2^n &=& \sum_{r = 0}^n \binom{n}{r}1^r1^{n-r}\\
\end{eqnarray*}
So 
\begin{eqnarray}
\sum_{r = 0}^n \binom{n}{r}  &=& 2^n \label{eqn:xpyistwo}
\end{eqnarray}

Pascal's triangle is an arrangement of the binomial coefficients in 
a table.  We will call the number in Row $n$ and Column $r$, 
$C(n,r)$. 
The table is easy to calculate.
Assume that all the values in Column $0$, $C(i, 0) = 1$, for 
$i = 1, 2, \ldots$ and the values in Column $1$ are
In column $1$ the values are given by $C(i, 1) = i$ for
$i = 1, 2, \ldots$.  This makes $C(1,0) = 1$ and $C(1,1) = 1$, 
Consider a portion of the table of numbers
\begin{table}[h]
\begin{tabular}{l|lll}
      & Columns         &                                   \\ 
Rows  & $r-1$          & $r$                               \\ 
\hline
$n-1$ & $C(n-1, r-1)$  & $C(n-1, r)$                       \\
$n$   & $C(n, r-1)$    & $C(n, r) = C(n-1,r-1) + C(n-1, r)$\\
\end{tabular}
\end{table}
of the binomial coefficients in a triangle. It is named after 
Blaise Pascal.  It has been studied for centuries and is still the 
focus of vigorous attention.

The rows of Pascal's triangle are conventionally enumerated 
starting with row zero, and the numbers in odd rows are 
usually staggered relative to the numbers in even rows. A 
simple construction of the triangle proceeds in the following 
manner. On the zeroth row, write only the number 1. Then, to 
construct the elements of following rows, add the number 
directly above and to the left with the number directly above 
and to the right to find the new value. If either the number 
to the right or left is not present, substitute a zero in its 
place. For example, the first number in the first row is 
0 + 1 = 1, whereas the numbers 1 and 3 in the third row are 
added to produce the number 4 in the fourth row."
\begin{tabular}{r|rrrrrrrrrrrr}
\small
$n$  & $\binom{n}{0}$& $\binom{n}{1}$& $\binom{n}{2}$& $\binom{n}{3}$& $\binom{n}{4}$& $\binom{n}{5}$& $\binom{n}{6}$& $\binom{n}{7}$& $\binom{n}{8}$& $\binom{n}{9}$& $\binom{n}{10}$& $\binom{n}{11}$\\
\hline
 0 & 1 &    &    &     &     &     &     &     &    &    &    &    \\      
 1 & 1 &  1 &    &     &     &     &     &     &    &    &    &    \\ 
 2 & 1 &  2 &  1 &     &     &     &     &     &    &    &    &    \\
 3 & 1 &  3 &  3 &   1 &     &     &     &     &    &    &    &    \\
 4 & 1 &  4 &  6 &   4 &   1 &     &     &     &    &    &    &    \\
 5 & 1 &  5 & 10 &  10 &   5 &   1 &     &     &    &    &    &    \\ 
 6 & 1 &  6 & 15 &  20 &  15 &   6 &   1 &     &    &    &    &    \\ 
 7 & 1 &  7 & 21 &  35 &  35 &  21 &   7 &   1 &    &    &    &    \\
 8 & 1 &  8 & 28 &  56 &  70 &  56 &  28 &   8 &  1 &    &    &    \\
 9 & 1 &  9 & 36 &  84 & 126 & 126 &  84 &  36 &  9 &  1 &    &    \\ 
10 & 1 & 10 & 45 & 120 & 210 & 252 & 210 & 120 & 45 & 10 &  1 &    \\ 
11 & 1 & 11 & 55 & 165 & 330 & 462 & 462 & 330 &165 & 55 & 11 &  1 \\ 
\hline                          
\end{tabular}
\startproblem{Display Pascal's triangle}

Design an algorithm that will display a Pascal triangle.  Use recursion 
to calculate the terms.

\[
b(n, k) = b(n-1, k-1) + b(n-1, k)
b(n, 0) = 1
b(n, n) = 1\]
\problemend

\startsolution{A recursive algorithm}

\noindent Recursive function to calculate terms

\begin{lstlisting}[keywordstyle=\texttt]]
function pascal(n,k)
  begin block
    if (k = 0) or (k = n) then return set pascal = 1
    else return set pascal = pascal(n-1, k-1) + pascal(n-1, k)
  end block
\end{lstlisting}
\solutionend


\startsolution{Binomial coefficients using addition}\label{sol:binomialadditions}
\solutionend


\startproblem{Display Pascal's triangle}
Pascal's triangle must be presented in a tabular form.
\problemend

\startproblem{Display Pascal's triangle as a triangle}
Display Pascal's triangle in the form
\begin{verbatim}
                             1
                          1     1
                       1     2     1
                    1     3     3     1
                 1     4     6     4     1
              1     5    10    10     5     1
           1     6    15    20    15     6     1
\end{verbatim}
\problemend
By convention the first row is called `Row $0$'.  The sum of its 
coefficients is $1$.  The sum of the coefficients of Row $4$ is $15$, 
in general the sum of the binomials in row $r$ is $2^r$.

\startsolution{Displaying Pascal's triangle}
\begin{lstlisting}[keywordstyle=\texttt]]
get num from user
for n ranges from 0 to num do
  begin block
    for k ranges from 0 to num do
       begin block
         display pascal(n, k)
       end block
  end block
\end{lstlisting}
\solutionend

\startproblem{Perfect numbers}
Design an algorithm that will check if a number supplied by the user 
is a perfect number.

\noindent A perfect number is a number, $n$,  such that:
\[n = f_1 + f_2 + \ldots + f_k\]
where the $f_i$, $i = 1, 2, \ldots, k$ are the factors of $n$.

In other words:  If a number is equal to the sum of all its factors, 
excluding itself, then that number is a perfect number.
\problemend

\noindent Algorithm

\startsolution{Perfect numbers}
This function finds all the factors of a number, places them in an 
array and returns the number of factors found.

\begin{lstlisting}[keywordstyle=\texttt]]
factors : array [1..1000] of numbers

function fact (num)
  begin block
    set cnt = 0;
    for i ranges from 1 to num do
      begin block
        if num is divisible by i then
          begin block
            set cnt = cnt + 1
            set factors[cnt] = i
          end block
      end block
    return set fact = cnt
  end block
\end{lstlisting}

\noindent The main algorithm

\begin{lstlisting}[keywordstyle=\texttt]]
get num from user
count = fact (num)
set total = 0
for i ranges from 1 to count - 1 do
   set total = total + factors(i)
if total = num then display num is a perfect number
\end{lstlisting}
\solutionend


\startproblem{Display the first 15 perfect numbers}
\problemend

\startsolution{Display the first 15 perfect numbers}
\begin{lstlisting}[keywordstyle=\texttt]]
set count = 0
set num = 1
repeat
  count = fact (num)
  set total = 0
  for i ranges from 1 to count - 1 do
    set total = total + factors(i)
  if total = num then 
    begin block
      set count = count + 1
      display num
    end block
  set num = num + 1
until count = 15
\end{lstlisting}
\solutionend

\startproblem{Prime numbers again}
Use an array to construct an efficient algorithm to find prime numbers.
\problemend

\startsolution{Prime numbers with an array}
We know that if $n$ is divisible by $d$, then $n$ must also be divisible 
by all of $d$'s factors, e.g.  $16$ is divisible by $8$.  It is also divisible 
by $8$'s factors, $1$, $2$, $4$.  This means that when we check for factors 
of a number we can ignore compound numbers and consider only the prime divisors 
of $n$.
\solutionend

\noindent Let's find the first 1000 prime numbers


Algorithm

\begin{lstlisting}[keywordstyle=\texttt]]
primes : array [1..1000] of numbers

set num_of_primes = 1
set primes[num_of_primes] = 2
set number_to_test = 3
repeat
  set cnt = 1
  while (num_to_test is not divisible by primes[cnt]) and (primes[cnt] < sqrt(num_to_test)) do
    begin block
      set cnt = cnt + 1
    end block
  if num_to_test is not divisible by primes[cnt] then
    begin block
      set num_of_primes = number_of_primes + 1
      set primes[number_of_primes] = num
    end block
  set num_to_test = num_to_test + 1
until num_of_primes = 1000
for i ranges from 1 to 1000 do
  display primes[i]
\end{lstlisting}

\startproblem{2}
Find all prime numbers between Num1 and Num2.
\problemend


\startsolution{Getting primes}

What is a prime number?
It is a number that is divisible only by itself and 1.

How would we check to see if a number is prime or not?
If a number is divisible by any number other than 1 or itself, then that number is not prime.

What do we know/have?
We have already constructed several algorithms to check whether a number is prime or not. All we need to know now is use one of these algorithms to check each number between Num1 and Num2.

Algorithm

In plain English

\begin{lstlisting}[keywordstyle=\texttt]]
Get Num1, the lower boundary of the range
Get Num2, the upper boundary of the range
For each number between Num1 and Num2
   Assume the number is prime
   Test divisibility of the number by all potential factors between 1 and the number itself
   If the number is divisible by any of these numbers, then it is not prime, otherwise it is prime
\end{lstlisting}


In pseudo code

\begin{lstlisting}[keywordstyle=\texttt]]
repeat loop

get Num1 from user
get Num2 from user
for Num ranges from Num1 to Num2 do
  begin block
    set PFactor = 1
    repeat
      set PFactor = PFactor + 1
    until (PFactor > square root of (Num)) or (Num divisible by PFactor)
    if PFactor <= square root of (Num) then display Num is not prime
    else display Num is prime
  end block
\end{lstlisting}


while loop

\begin{lstlisting}[keywordstyle=\texttt]]
get Num1 from user
get Num2 from user
for Num ranges from Num1 to Num2 do
  begin block
    set PFactor = 2
    while (PFactor < square root of (Num)) and (Num not divisible by PFactor)
      begin block
        set PFactor = PFactor + 1
      end block
    if PFactor <= square root of (Num) then display Num is not prime
    else display Num is prime
  end block University of the Western Cape
\end{lstlisting}
\solutionend

\chapter{A part of James' notes in their original format}
Computer Science I
Problem Solving

James Connan
jconnan@uwc.ac.za


Academic Responsibility

Academic dishonesty will not be tolerated.  Students suspected of academic dishonesty will be referred to the Proctor's office.


Course Objectives

Read, understand and solve problems\\
Problem solving techniques\\
Understanding of sequence, selection and repetition control structures\\
Design algorithms to solve problems\\
Express algorithms in pseudo code or structured diagrams\\
Improved analytical thinking and problem solving skills\\


What does Computer Scientists do?

Computer Scientists use computers to solve problems\\
Often new software needs to be developed to solve problems\\
Software development requires problem solving and programming skills\\


What is Problem Solving?

Problem Solving is a process\\
The first step in Problem Solving is understanding and exploring the problem\\
Next a strategy for solving the problem must be found\\
Use the strategy to solve the problem\\
Reflect on the solution\\


What are Algorithms?

Algorithms describe methods by which tasks are to be accomplished\\
An Algorithm is a set of instructions\\
When an Algorithm is executed a task is accomplished\\


Example Algorithms

Knitting pattern (e.g. Jersey)\\
Assembly Instructions (e.g. Models, furniture, etc.)\\
Recipe (e.g. Cake)\\
Pattern (e.g. Dress, jacket, etc.)\\
Musical Score (e.g. Beethoven, Bach, etc.)


Syntax, Semantics and Logic

When designing algorithms there are certain common errors that must be avoided.  These are errors with syntax, semantic and logic. Syntax rules state how symbols in a language are used.  In the sentence 'sleeve the seams' or the expression '2+=3' valid English and mathematical symbols are used, but the way they are combined does not result in a valid sentence or expression.\\
Semantic rules examine the meaning of symbols.  The sentence 'the elephant ate the peanut' is a valid English sentence, but 'the peanut ate the elephant' makes no sense.\\
Logic rules check if the solution properly describes the process.  If an incorrect formula is used or an invalid conclusion is made then the algorithm will fail.


Stepwise/Top-down Refinement of Algorithms

When constructing an algorithm, it is a good idea to start of by describing the solution in the most general terms.  So, designing an algorithm to make a cup of coffee, the first iteration yields:

1.boil water\\
2.put coffee in cup\\
3.add water to cup\\

Each step is now examined and were sensible further refinements are made.

1.boiled water

This can be further refined by saying:

  1.1.fill kettle\\
  1.2.switch on kettle\\
  1.3.wait until boil\\
  1.4.switch off kettle\\


2.put coffee in cup

Can be expanded to:

  2.1.open coffee jar\\
  2.2.measure coffee\\
  2.3.put coffee in mug\\
  2.4.close jar


3.add water to cup

Becomes

  3.1.pour water from kettle into mug until mug full

Again the algorithm is checked to see if there are any steps that need further refinement.

1.boil water\\
  1.1.fill kettle\\
    1.1.1.put kettle under tap\\
    1.1.2.turn on tap\\
    1.1.3.wait until kettle full\\
    1.1.4.turn off tap\\
  1.2.switch on kettle\\
  1.3.wait until boil\\
    1.3.1.wait until kettle whistles\\
  1.4.switch off kettle\\
2.put coffee in cup\\
  2.1.open coffee jar\\
    2.1.1.take jar off shelf\\
    2.1.2.remove lid\\
  2.2.measure coffee\\
  2.3.put coffee in mug\\
  2.4.close jar\\
    2.4.1.replace lid\\
    2.4.2.replace jar on shelf\\
3.add water to cup\\
  3.1.pour water from kettle into mug until mug full\\

There are no now further sensible refinements that can be made to the algorithm, so the algorithm is complete.

Sequential Algorithms

The coffee-making algorithm is an example of a sequential algorithm.\\
Sequential algorithms have the following characteristics:\\
1.Steps are executed one at a time\\
2.Each step is executed exactly once: none are repeated or omitted\\
3.The steps are executed in the same order in which they are written\\
4.Termination of the last step implies termination of the algorithm\\



Trace Tables

Trace tables are used to verify that the algorithm performs the desired task. All variables are written down in such a way as to serve as headers for columns that will be used to track them as the algorithm is executed.  The algorithm is then executed one line at a time and changes to all variables are noted.  It may be desirable to keep track of any output produced by the algorithm as well.

Conditionals

Sequential algorithms allows us to solve many problems, but sometimes the solution to an problem can not be expressed in a sequential way.  There may also be instructions that we only want to execute under certain conditions.  In the coffee making algorithm, for example, if the kettle is already full we do not need to fill it.  The conditional statements we have are as follows:
   
If condition then statement\\
The statement will only be executed if the condition holds.

If condition then statement1 else statement2\\
If the condition holds statement1 is executed and if it does not hold statement2 is executed.
 
Case variable of \\
  condition1 : statement1\\
  condition2 : statement2\\
  condition3 : statement3\\
Case statements can be used to replace multiple if statements.  If the variable matches condition1 statement1 will be executed.  Same condition2 etc.



Looping Constructs

Solutions often include some form of repetition.  The same action is repeated over and over with few or no changes.  If the objective is to work out the average of some numbers then the number must be read in and totalled one at a time until all numbers have been added.  The total is then divided by the number of numbers to obtain the average.  There are three looping structures available to us.  These are the for-loop, the while-loop and the repeat-loop.

The for-loop
 
for variable ranges from start value to end value do\\
  statement

The value of variable is set to start value.  The statement is then executed.  The value of variable is then incremented and statement is executed again.  This is repeated until the value of variable exceeds end value.  At this stage the loop exits.  It is also possible to set start value higher than end value and decrement variable each iteration until it reaches end value. 

The while-loop

while condition execute do\\
  statement

As long as the condition is valid the while-loop will continue executing the statement.  This means that statement must change something to allow condition to become untrue or the loop will never terminate.  There are examples where it is desirable to have a loop that never terminates, but in most cases this is not true.

The repeat-loop

repeat \\
  statement\\ 
until condition

The statement will be executed until the condition is met.  If the statement does not change something to allow the condition to become true then the loop will never terminate.

Some rules of thumb for loops\\
1.If we know exactly how many times we want a loop to be executed, it is normally desirable to use a for-loop.\\
2.The main difference between a while-loop and repeat-loop is that with a while-loop the condition is checked before the loop is entered while with a repeat-loop the condition is checked after the first iteration of the loop.\\
3.repeat-loops and while-loops are usually interchangeable.  In order to change from the one to the other the condition needs to be negated.  \\


Statement Blocks

So far we have only referred to single statements.  If we have multiple statements we can encapsulate them with a begin block and end block.  These statements then become a single unit.

start block\\
  statement\\
  statement \\
  statement \\
end block\\


begin ... end
{ ... }

Nested Loops

Loops within loops


Arrays

A lot of problems requires the use of lists as part of the solution. One way to handle lists is to use arrays.  The syntax is as follows:

arrname : array [start..end] of type 

where,
arrname is the name of the array,\\
start..end is the range of the array,\\
and type is the type of data that can be stored in the array

Examples

\begin{lstlisting}
num : array [1..3] of number

          1 2 3
          - - -
num -->  | | | |
          - - - 
\end{lstlisting}

This array provides us with a variable, num, that has three 'slots' of type number in which we can store values.  These 'slots' are accessed by using num[1], num[2] and num[3].\\

\begin{lstlisting}
\end{lstlisting}
word : array[1..15] of character\\

           1  2  3  4  5  6  7  8  9  10 11 12 13 14 15\\
           -- -- -- -- -- -- -- -- -- -- -- -- -- -- --\\
word -->  |  |  |  |  |  |  |  |  |  |  |  |  |  |  |  |\\
           -- -- -- -- -- -- -- -- -- -- -- -- -- -- --\\

This array provides us with 15 'slots' of type character.  They are accessed by using word[1], word[2], ... , word[15]\\

\begin{lstlisting}
\end{lstlisting}
num : array [5..8] of number

          5 6 7 8\\
          - - - -\\
num -->  | | | | |\\
          - - - -\\

This provides us with four 'slots' referenced using num[5], num[6], num[7] and num[8]

set arr[2] = 7

This will store the value 7 in the position two of the array.

set arr[i] = 5

This will store the value 5 in position i of array arr.  The value i must be within the range of the array.  That means that the value of i must be between start and end.

set arr[i+1] = 5 

Same as above, except that i+1 must now be within the range of the array.

set arr[5] = arr[7]

This will set the value of the 5'th element of the array equal to the value of the 7'th element.

set num = arr[3]

This will set the value of num equal to the value stored in position 3 of array arr.

Arrays can have more than one dimensions.  That is:

\begin{lstlisting}
\end{lstlisting}
num : array [1..3][1..5] of number


            1 2 3 4 5\\
            - - - - -\\
num -->  1 | | | | | |\\
            - - - - -\\
         2 | | | | | |\\
            - - - - - \\
         3 | | | | | |\\
            - - - - -\\

This visualization is often used to make it easier to understand what multiple (in this case 2) dimensional arrays are.  We now have 15 'slots' and each 'slot' is identified by two numbers.  num[1,1] is the first element in the array and num [3,5] is the last.  There are now two indexes that need to be used to reference each element and each must be within the boundaries defined for its range.  It is important to remember that the visualization above is exactly that, just a visualization.  The array could also have been visualized as follows:

\begin{lstlisting}
\end{lstlisting}
          1,1 1,2 1,3 1,4 1,5 2,1 2,2 2,3 2,4 2,5 3,1 3,2 3,3 3,4 3,5\\
          --- --- --- --- --- --- --- --- --- --- --- --- --- --- ---\\
num -->  |   |   |   |   |   |   |   |   |   |   |   |   |   |   |   |\\
          --- --- --- --- --- --- --- --- --- --- --- --- --- --- ---\\

We are not restricted to two dimensions.

num : array [1..3][4..6][1..5][1..2][3..5] of number

This is a five dimensional array.  We can store 3x3x5x2x3 = 270 elements in this array.  Each element is referred to by 5 indexes.  Hence, num [i,j,k,l,m] refers to one element in the array.  Each of the indexes i,j,k,l,m must be within its defined range.

Arrays, combined with the looping structures we have already encountered, are very powerful tools to help us solve problems.

Functions and Procedures

We have already encountered block statements.  When we have a block of statements that perform a particular task it is often convenient to label this block in a special way that allows us to refer to it again at a later time.  For example:

The following algorithm calculates and displays x to the power of y.

\begin{lstlisting}
\end{lstlisting}
set ans = 1\\
for i ranges 1 to y do\\
  set ans = ans * x\\
display ans\\

We can rewrite this as\\

\begin{lstlisting}
\end{lstlisting}
procedure pow(x,y)\\
begin block\\
  set ans = 1\\
  for i ranges from 1 to y do\\
    set ans = ans * x\\
  display ans\\
end block\\

We can now call this procedure from another algorithm

\begin{lstlisting}
\end{lstlisting}
get num from user\\
get power from user\\
pow (num,power)\\

We can also write this as a function\\

\begin{lstlisting}
\end{lstlisting}
function pow(x,y)\\
begin block\\
  set ans = 1\\
  for i ranges from 1 to y do\\
    set ans = ans * x\\
  return set pow = ans\\
end block\\

The difference between a procedure and function is that a procedure can only act on data/variables while a function can return a value.  

\begin{lstlisting}
\end{lstlisting}
get num from user\\
get power from user\\
set result = pow(x,y)\\
display result\\

Recursion

A function or procedure is said to be recursive when it calls itself.  It is possible to rewrite the power function above as a recursive function.

\begin{lstlisting}
\end{lstlisting}
function pow(x,y)\\
begin block\\
  if y = 0 then return set pow = 1\\
  else return set pow = pow(x,y-1) * x\\
end block\\

When writing a recursive function the first thing to do is to find the base case. For example:  

Write an algorithm to work out n!

1! = 1\\
2! = 1 x 2\\
3! = 1 x 2 x 3\\
4! = 1 x 2 x 3 x 4\\
etc

we can rewrite this as 

1! = 1\\
2! = 1! x 2\\
3! = 2! x 3\\
4! = 3! x 4\\
etc

in general \\

n! = (n-1)! x n\\

the base case is \\
\begin{lstlisting}
\end{lstlisting}
when n = 1 then n! = 1\\

Thus

\begin{lstlisting}
function fact(n)
begin block
  if n = 1 then return set fact = 1
  else return set fact = fact (n-1) x n
end block
\end{lstlisting}

Write a recursive algorithm that will add two numbers.Roots


\startproblem{Roots of a quadratic equation}
Write an algorithm that will find and classify the roots of the quadratic equation
\[ax^2 +bx + c = 0\]
then 
\[x = (-b +- sqrt(b^2 - 4ac))/2a\]
\problemend


Notes:

if $a = 0$ then $x = -c/b$\\
if $b = 0$ then $x = \sqrt(-4ac)/2a$\\
if $c = 0$ then $x = -b/a$\\

We need to ensure that $\sqrt x$ exists, i.e. that the \textit{argument}, 
of the square root function $x$ is positive.

Remember that the square root of something is always non negative, 
in other words it is greater or equal to zero.\footnote{Don't let this definition of 
square root mislead you: \texttt{The square root of a number is that number that 
when it is squared gives the number.} Since the $(-10)^2$ is $100$ it leads you to believe that 
the square root of $100$ is $-10$ or $10$.  The $\sqrt{100} = 10$, it is not $-10$.}


Algorithm

\begin{lstlisting}[keywordstyle=\texttt]]
get a from user
get b from user
get c from user

if a = 0 then
  begin block
    if c <> 0 then display x = -b/c
    else 
       begin block
         if b <> 0 then display x = 0
         else display a=b=c=0
       end block
  end block
if a <> 0 then 
  begin block
    if b^2 > -4ac then display x = (-b +- sqrt(b^2 - 4ac))/2a
    else display roots are not real
  end block
\end{lstlisting}

Write a program that searches through an array 
sequentially and returns the index of the desired 
value and the number of lookups made.Write a program that will find the sum of the first 
n terms of a Tailor series.Write a program that will convert a temperature from 
Celsius to Fahrenheit and vice versa.Write a program that generates n unique random numbers.

\startproblem{Make a list of $10$ different random numbers}
Design an algorithm to will construct an array of 10 unique random whole numbers.
\problemend

\startsolution{Make a list of $10$ different random numbers}
\begin{lstlisting}[keywordstyle=\texttt]]
numbers : array [1..10] of number

function insert (num, cnt)
  begin block
    set found = false
    for i ranges from 1 to cnt do
      if numbers[i] = num then set found = true
    if found <> true then
      begin block
        set cnt = cnt + 1
        set numbers[cnt] = num
      end block
    return set insert = cnt
  end block  
\end{lstlisting}

\begin{lstlisting}[keywordstyle=\texttt]]
set cnt = 0
repeat
  set x = random
  cnt = insert(num, cnt)
until cnt = 10

Write a program tat will allow the user to calculate the sum, 
dot (http://en.wikipedia.org/wiki/Dot_product) product 
or cross product (http://en.wikipedia.org/wiki/Dot_product) of tho vectors.
\end{lstlisting}
\solutionend

\startproblem{Create a list of $10$ random integers in a tighter  range}
Now redesign your the algorithm to construct an array of 10 unique 
random whole numbers in the range $[A, B)$, for example, let the range
be numbers from the set $\{17, 18, 19, 20, 21, 22, 23, 24, 25, 26, 
27, 28\}$, represented by the range $[17, 29)$.  Note that (1) the 
range excludes $29$, and that there are enough different numbers to 
enable the construction of at least $10$ different numbers.
\problemend

\startproblem{Create a list of $10$ random integers}
Now redesign your the algorithm to select $10\%$ of the numbers from 
an array of $N$ unique random whole numbers in the range $[A, B)$, 
for example, let the range be numbers from the set $\{1, 2, 3, \ldots, 100\}$ 
denoted by the range $[1, 101)$.  Note that (1) the range excludes 
$101$, and that there are enough different numbers to enable the construction 
of at least $10\%$ of the numbers.
\problemend

\startproblem{Water flow}
Design an algorithm that will perform water flow rate calculations given that

\begin{tabular}{|c|c|}
\hline
Volume of water (kl) &   Cost (R/kl)\\
\hline
0 to 80.0   &   0.90\\
80.1 to 120.0 &     1.15\\
120.1 or more  &    1.25\\
\hline
\end{tabular}
\problemend

Algorithm

\begin{lstlisting}[keywordstyle=\texttt]]
get volume from user
if volume > 120 then set cost = (volume-120)x1.25 + 40x1.15 + 80x0.90
if volume > 80 and volume <=120 then set cost = (volume-80)x1.15 + 80x0.90
if volume <= 80 then set cost = volume x 0.90
display volume water costs cost
\end{lstlisting}
\section{Dates and days}
\label{section:dates}
Calculations with dates and clocks are common applications of modular 
arithmetic.

\startproblem{ Given the year, calculate if it is a leap year or not.}
If the year is 
divisible by $400$ it is a leap year.  If it is divisible by $100$ and not by 
$400$ it is not a leap year.  The years $1700$, $1800$ and $1900$ 
are not leap years, but $1600$, and $2000$ are leap years.  Of the 
remaining years, every year that is not divisible by $100$ but is divisible 
by $4$ is a leap year.  For example $2008$, $2012$, $2016$ are leap years.
\problemend

\startsolution{Calculating leap years}
The leap year status is determined by some \lstinline!if! statements.
\begin{lstlisting}
def isLeapyear(year):
  if year % 400 == 0:
    return True
  if year % 100 == 0:
    return False
  if year % 4 == 0:
    return True
  else:
    return False
\end{lstlisting}
A shorter version is:
\begin{lstlisting}
def isLeapyear(year):
  return year % 400 == 0 or year % 4 == 0 and year % 100 != 0
\end{lstlisting}
\solutionend

\startproblem{Number of days in the months}
\textit{Calculate} the number of days in the months 
of a given the \textit{year}.
\problemend

\startsolution{Days in a month}
\begin{lstlisting}
daysinmonth = [30 + (month + month//8 ) % 2 for month in range(1, 13)]
\end{lslisting}
gives
\begin{lslisting}
[31, 30, 31, 30, 31, 30, 31, 31, 30, 31, 30, 31]
\end{lstlisting}
All that still needs to be done is to adjust for February.  What we 
do is to replace \lstinline!daysinmonth[1]! by 
 \lstinline!28 + int(isLeapyear(year))].  The \lstinline!int(True)! is $1$ and
\lstinline!int(False)! is $0$.  The complete code for the days in the months follows.
\begin{lstlisting}
def daysInmonth(year):
  days = [30 + (month + month//8 ) % 2 for month in range(1, 13)]
  days[1] = 28 + int(isLeapyear(year))
  return days
\end{lstlisting}
\solutionend

\startproblem{Day of the week}
A popular application of \textit{modulo} arithmetic concerns calculating the 
day of the week.  Given the Gregorian calendar date determine
the day of the week.
\problemend


\startsolution{Day of the week}
One start by counting the difference in days by calculating the number 
of years, months and days from a base date for which you know the 
day of the week, and then simply getting the day of the week using 
\textit{modulo} $7$.  We call this number the \textit{daydate}.

It does not matter what the starting date is, if the days counted 
are correct then calculate daydate for today and get 
\lstinline!today = daydate % 7!.  Since we know what day of the week
today is, henceforth \lstinline!today! represents that day of the week.
Suppose today is a Tuesday, and \lstinline!today!$= 3$, then $3$ is a
Tuesday, $4$ is a Wednesday, $5$ is a Thursday, $6$ is a Friday, $0$ 
is a Saturday, $1$ is a Sunday, $0$ is a Monday.  Now we calculate 
daydate, given the parameters \textit{year}, \textit{month}, \textit{day}

\begin{lstlisting}
def daydate(year, month, day):
  mdays = [30 + (m + m//8 ) % 2 for m in range(1, 13)]
  mdays[1] = 28 + int(isLeapyear(year))
  days = 1
  mdays = [0]+mdays[:-1]

  for m in range(month):
    days += mdays[m]
    mdays[m] = days

  days = day + mdays[month-1]
  years = year - 1600   # Our base date is 1601-01-01
  days += years * 365
  leapcenturies = ((years - 1) // 100 - 16)//4 + 1 # Add leap years
  days += leapcenturies 
  weekdays = ["Mon", "Tue", "Wed", "Thu", "Fri", "Sat", "Sun"]
  return weekdays[days % 7]
\end{lstlisting}
\solutionend


\startproblem{ Calculate the Gregorian calendar date of Easter Sunday.}
The rules for determining the date of Easter or Passover are quite 
easy.  During 325 ACE in Nicaea the first ecumenical council of the 
catholic church determined the date of Easter as the first Sunday 
after the full moon after March 21---the spring equinox in the Northern 
hemisphere.  The calculations were based on the Julian calendar and 
since $1582$ and $1752$ in the United Kingdom and the United States 
of America on the Gregorian calendar. The date of Easter varies between 
March 22 and April 25. 
\problemend

\startsolution{The Gauss 1812 algorithm for calculating Easter}
\noindent{\Large\textbf{The algorithm of Gauss for calculating Easter}}
\begin{lstlisting}

a = year %19
b = year % 4
c = year % 7
k = floor (year/100)
p = floor ((13 + 8k)/25)
q = floor (k/4)
M = (15 − p + k − q) % 30
N = (4 + k − q) % 7
d = (19a + M) % 30
e = (2b + 4c + 6d + N) % 7
Gregorian Easter is 22 + d + e March or d + e − 9 April
if d = 29 and e = 6, replace 26 April with 19 April
if d = 28, e = 6, and (11M + 11) % 30 < 19, replace 25 April with 18 April
For the Julian Easter in the Julian calendar M = 15 and N = 6 (k, p, and q are unnecessary)
\end{lstlisting}
\solutionend


\startsolution{Easter in a spreadsheet}
The following %are two 
spreadsheet 
formula %s 
calculates the date of Easter Sunday.
%#\begin{lstlisting}
%#=DOLLAR(("4/"&A1)/7+MOD(19*MOD(A1,19)-7,30)*14%,)*7-6
%#\end{lstlisting

\begin{lstlisting}
=ROUND(DATE(A1,4,1)/7 + MOD(19*MOD(A1,19) - 7, 30)*14%, 0)*7 - 6
\end{lstlisting}
\solutionend

\startproblem{Birthdays on the same day}
Suppose that a room is occupied by some people.  If there is only 
one person in the room nobody will share his birthday so the probability 
$q$ that there is nobody to share his birthday is $q_1=\frac{365}{365} = 1$. 
 When there are two people in the room the probability that no two 
people share a birthday becomes $q_2=\frac{365\times364}{365^2}= 0.99\overline{72602739}$.\footnote{Notice 
how we write the repeating decimal $0.99\ 72602739\ 72602739\ldots$.}
In general if there are $n$ people in the room the probability that 
no pair of people share a birthday becomes
\[q_n=\frac{\prod_{r=1}^n 365-r+1}{365^n}.\]
The problem is to discover how many people must be in the room for 
the probability that there are two people that have a birthday on 
the same day are in the room, i.e. find an $n$ such that $p = 1-q_n > 0.5$.
\problemend
\startsolution {Birthdays on the same day}
Simply calculate the running fraction $q_n$ to discover at which $n$ 
it becomes less that a half.
\solutionend

\chapter{Problems with solutions in python}
\section{}
\label{section:introduction}
\startproblem{$S = \sum_{r=1}^Nr$}
Suppose that the integer $N$ is given.  Construct an algorithm to add 
all the integers from 1 up to and including $N$, i.e. calculate
the sum $S$,\[S = \sum_{r=1}^Nr = 1+2+\ldots+N.\]
\problemend

\startsolution{$S = \sum_{r=1}^Nr$}
We first initialize a variable called \texttt{S} to zero and subsequently
add the values of \texttt{r} running from $1$ to $N$ one by one.

\begin{lstlisting}
N = 10
S = 0
for r in range(1, N+1):
  S += r
print ("Sum of integers from 1 to %d is %d" % (N, total))
\end{lstlisting}

A much tidier solution, more in the spirit of Python style, is:

\begin{lstlisting}
N = 10; S = sum([r for r in range(1, N+1)])
\end{lstlisting}
The \lstinline!for r in range(1,N+1)! generates each \lstinline!r! which 
is then agglomerated into a list by the surrounding brackets \lstinline![!\dots\lstinline!]!.
The function \lstinline!sum! is built in, and yields the sum of the numbers 
in a list.

If you were Carl-Fried Gauss, at the age of about ten years, you 
would no doubt have noticed that the sums 
\[1, 2, \ldots, N-1, N\]
\[+\]
\[N, N-1,  \ldots, 2, 1\]
add up to 
\[N+1, N+1, \ldots, N+1,\]
and that half of this is the required sum, and further you 
would have seen that there are $N$ of these $N+1$ terms and you have 
given the answer in a flash as
\[\frac{N\times (N+1)}{2}.\]

Gauss would have programmed the requested result as

\begin{lstlisting}[keywordstyle=\texttt]
N = 10; S = N*(N+1)/2
\end{lstlisting}

Notice that despite the division by two, the result is a whole number.
\solutionend

\startproblem{Sum a list of integers}
Suppose an array of numbers is given in the list $L$, e.g.
\begin{lstlisting}[keywordstyle=\texttt]
L = [17, 23, 29, 31, 37, 41, 47]
\end{lstlisting}
Sum the numbers in the list.  You may assume that all the entries 
in the list are integers.
\problemend

\startsolution{Sum the numbers in a list}
Suppose $L$ is ready to use, then the simple Python answer is

\begin{lstlisting}
S = sum(L)
\end{lstlisting}
\solutionend

\startproblem{Sum numbers stored in a file}
On the other hand it is more likely that the numbers reside in some
file, ``with each number on its own line'',\footnote{In actual fact it 
means that the file has each integer terminated by the newline character 
`\lstinline!\\n!'. So the list in our example is stored in a file as 
the single string \lstinline!"17\\n23\\n29\\n31\\n37\\n41\\n47\\n51\\n53\\n"!. 
Check that the file has altogether 27 characters.  The file does not 
end with a character for the end of the file.} and where the file 
is displayed by an editor as:
\begin{lstlisting}
17
23
29
31
37
41
47
\end{lstlisting}
\problemend

\startsolution{Sum numbers stored in a file}
Let us now read the list of numbers and add them one by one.
\begin{lstlisting}
f = open ("listofintegers.text", 'r')
L = f.readlines()
f.close()
S = 0
for r in L:
  S += int(r)
print ("Sum of integers in the list is %d" % S)
\end{lstlisting}
The \lstinline!f.readlines()! bunches the substrings separated by 
newline characters together into one string.  This string still has 
to be converted into an integer by the function \lstinline!int(r)! 
before it can be summed.
\solutionend

\startsolution{Sum numbers stored in a file---Python style}
A more Pythonic way to program this follows
\begin{lstlisting}
print ("Sum of integers in the list is %d" % 
        sum([int(r) for r in open("lines.text",'r').readlines()]) )
\end{lstlisting}
Since the number $r$ is read in as a string from the file we turn 
it into an integer using the built in \texttt{int} function so that 
it may be summed.
\solutionend

\startproblem{Sum numbers recursively}
The next version of the $\sum_{r=1}^Nr$ problem sets it in a new angle.
Sum the numbers $1$ to $N$ using a \textit{recursive} function called 
by invoking \lstinline!add(N)!.  A recursive function is defined in 
terms of itself.
\problemend

\startsolution{Sum numbers recursively}
The idea is that when called as \lstinline!add(1)!, the function 
yields the sum $1$.  When \lstinline!add(N)! is called with a value of 
$N$ exceeding $1$, the result is defined as the sum of that value of $N$
added to the value of \lstinline!add(N-1)!.  At each step the function 
calls itself, but with an argument that is smaller by 1.  These contractions
of the arguments yield what is known as a \textit{contraction mapping}
which eventually turns the argument into $1$ which will then be added 
to $2$, \ldots until finally $N$ is added.  So, first test if $N$ 
is $1$ and otherwise return $N + add(N-1)$.
\begin{lstlisting}
def add(N):
  if N == 1:
    return N
  else:
    return N + add(N-1)

print (add(10))
\end{lstlisting}
\solutionend

\startproblem{Add using \lstinline!inc! and \lstinline!dec!}
Add the two positive integer arguments $A$ and $B$ with \lstinline!add(A,B)! 
given the two functions \lstinline!inc(N)! which returns  $N + 1$ 
and \lstinline!dec(N)! which returns the value of $N-1$.
\problemend

\startsolution{Add using \lstinline!inc! and \lstinline!dec!}
The idea is to add $B$ to $A$ in the following manner.  First check 
if $B$ is zero, in which case the value of $A$ is returned.   Otherwise 
return the value of the \lstinline!add! of $A$ increased by $1$, and 
$B$ decreased by $1$.  In other words return the value of 
\lstinline!add(inc(A), dec(B))!.  This continues until the second 
argument becomes a zero when the value of the first argument is 
returned.
\begin{lstlisting}
def add(A, B):
  if B == 0:
    return A
  else:
    return add(inc(A), dec(B))

print (add(3, 7))
\end{lstlisting}
Notice how the second argument is contracted until it becomes zero, 
at which stage the first argument has been incremented by one enough 
times to have reached $A+B$.
\solutionend

\startproblem{Multiply without $\times$-operator}
Show how to multiply two positive integers numbers using addition only.
The function \lstinline!mult(A, B)! must yield the value of $A\times B$.
\problemend

\startsolution{Multiply with addition}
The idea is to add the value of $A$ repeatedly until it has been added 
$B$ times.  The function returns zero when the value of $B$ is zero,
and returns the value of $A$ when $B$ is one, otherwise the function
returns \lstinline!mult(A, B - 1) + A!.  We have multiplied without using
addition. The Python code follows
\begin{lstlisting}
def mult (A, B):
  if B == 0:
    return 0
  elif B == 1:
    return A
  else:
    return add(mult(A, dec(B)), A)

print (mult(3,7))
\end{lstlisting}
\solutionend

\startproblem{Recursive exponentiation}
Show how to exponentiate a positive integer to a positive integer power 
using multiplication only.
\problemend

\startsolution{Recursive exponentiation}
We will call the function \lstinline!power(A, B)! and it will yield 
the value of $A^B$ which is $A\times A\times A \ldots A$, $B$ times.
The idea is to multiply the value of $A$ repeatedly until it has been multiplied 
$B$ times.  The function returns $1$ when the value of $B$ is zero,
and returns the value of $A$ when $B$ is one, otherwise the function
returns \lstinline!power(A, B - 1) * A!.  We have raised $A$ to the power $B$ 
using multiplication. The Python code follows
\begin{lstlisting}
def power (A, B):
  if B == 0:
    return 1
  elif B == 1:
    return A
  else:
    return mult(power(A, dec(B)), A)

print (power(3,7))
\end{lstlisting}
\solutionend
These recursive solutions have practical bounds---they just do not 
work on a computer with a small memory stack when the arguments get large.  
Our \lstinline!power! function will soon run out of memory on even the largest 
of machines.

\startproblem{More efficient exponentiation}
Calculate \lstinline!power(A, B)!$ = A^B$ more efficiently.
\problemend

\startsolution{Using logarithms to calculate exponents}
 The simple answer is to use logarithms.  Since $\log A^B = B\log A$, 
we write $A ^ B = \exp(B \log A)$ and get the result quite quickly. 
 The snag is that the accuracy of the $\log$ and $\exp$ is usually 
limited on computers to about $7$ digits for single precision or to 
$18$ digits for double precision---Python uses double precision reals. 
This may be a problem when doing exponentiation in cryptographic calculations 
where numbers with precisions of $256$ or $512$ binary digits are 
used.\footnote{Since $\lceil\log_{10} 2^512 / \log_{10}\rceil = 107$, 
 $512$ bits occupy $107$ decimal digits.} We give a version of this 
function
\begin{lstlisting}
def power (A, B):
  from math import log, exp
  return exp(B*log(A))

print (power(3, 100))
\end{lstlisting}
Python has a built-in operator for exponentiation, so we could simply have programmed
\begin{lstlisting}
print (3**100)
\end{lstlisting}
Python also has a built-in function for exponentiation, so the following is also a good way
\begin{lstlisting}
print (pow(3,100))
\end{lstlisting}
\solutionend

\startproblem{An efficient method for exponentiation}\label{fastexponentiation}
What is an efficient way of doing multiple digit exponentiation 
in Python, without the advantage of the built-in operator or function 
called \lstinline!pow(A, B)!?
\problemend

\startsolution{A fast method for exponentiation}
There are several fast methods.  We will show a fast method that 
uses an interesting trick for exponentiating $A^B$ where both numbers 
are positive integers.  We must calculate\\
$A^B = A\times A\times\ldots \times A$,  $B$ times, requiring $B-1$ multiplications.
Suppose we were asked to calculate $A^{32}$. This could have been done 
in less multiplications by 
\begin{eqnarray*}
A^2    &=& A\times A, \\
A^4    &=& A^2\times A^2,\\
A^8    &=& A^4\times A^4,\\
A^{16} &=& A^8\times A^8,\\
A^{32} &=& A^{16}\times A^{16}.\\
\end{eqnarray*}
Instead of $31$ multiplications $5$ are sufficient.   What do we do when $B$ is not a 
power of $2$?  If $B = 33$ the sequence of multiplications becomes
\begin{eqnarray*}
A^2    &=& A\times A, \\
A^4    &=& A^2\times A^2,\\
A^8    &=& A^4\times A^4,\\
A^{16} &=& A^8\times A^8,\\
A^{32} &=& A^{16}\times A^{16}.
\end{eqnarray*} 
At some stage one more multiplication by $A$ is needed, since $A^{33} = A^{32} \times A$.
If $B$ were $34$ then one more multiplication by $A^2$ is needed.

\noindent The table shows the number of multiplications needed for calculating some powers of $A$\\
\begin{table}
\caption{Number of multiplications needed for $A^n$.
Note: Table has not been checked for correctness.}
\begin{center}\begin{tabular}{|c|c|l|}
\hline
Power of & Number of       &           \\
 $A$     & multiplications &   How\\
\hline
$A^2$     &      1               &  $ A\times A               $     \\
$A^3$     &      2               &  $ A\times A  \times A     $     \\
$A^4$     &      2               &  $ A^2\times A^2           $     \\
$A^5$     &      3               &  $ A^2\times A^2 \times A  $     \\
$A^6$     &      3               &  $ A^2\times A^2 \times A^2$     \\
$A^7$     &      2               &  $ A^2\times A^2 \times A^3$     \\
$A^8$     &      3               &  $ A^4\times A^4           $     \\
$A^9$     &      4               &  $ A^4\times A^4 \times A  $     \\
$A^{10}$  &      4               &  $ A^4\times A^4 \times A^2$     \\
$A^{11}$  &      5               &  $ A^4\times A^4 \times A^3$     \\
$A^{12}$  &      4               &  $ A^4\times A^4 \times A^4$     \\
$A^{13}$  &      4               &  $ A^4\times A^4 \times A^5$     \\
          &   \vdots             &  $ A^4\times A^4 \times A^7$     \\
$A^{16}$  &      4               &  $ A^8\times A^8           $     \\
$A^{32}$  &      5               &  $ A^{16}\times A^{16}     $     \\
$A^{33}$  &      6               &  $ A^{32}\times A          $     \\
$A^{34}$  &      6               &  $ A^{32}\times A^2        $     \\
\hline
\end{tabular}
\end{center}
\end{table}
The power is calculated as follows:  start with a \lstinline!product! of
$1$.  Set \lstinline!mul! to \lstinline!A!.  If $B$ is odd then multiply
\lstinline!product! by \lstinline!mul! and replace $B$ by $B//2$.  Square 
and replace \lstinline!mul *= mul!.  Stop when $B = 0$.
\begin{lstlisting}
def power (A, B):
  mul = A
  product = 1
  while B != 0:
    if B % 2 == 1:
      product *= mul
    mul *= mul
    B //= 2
  return product

print (power(3,7))
\end{lstlisting}
This is a very efficient algorithm but Python's built-in \lstinline!pow(A, B)! 
has a much better performance.  How does Python do it?
\solutionend



\startproblem{$N!$ recursively}
Calculate the factorial function, $N! = 1\times 2\times 3\ldots\times N$
using recursion.
\problemend

\startsolution{$N!$ recursively}
When the argument of \lstinline!factorial(N)! is $1$ or $0$ return a $1$ 
otherwise the factorial of $N$ is \lstinline!N * factorial(N-1)!.
\begin{lstlisting}
def factorial (N):
  if N < 2:
    return 1
  else:
    return N * factorial(N-1)

print (factorial(6))
\end{lstlisting}
\solutionend

\startproblem{$N!$ iteratively}
Calculate the factorial function, $N! = 1\times 2\times 3\ldots\times N$ 
without using recursion.
\problemend

\startsolution{$N!$ iteratively}
The non-recursive solution of $N!$ applies the definition directly.
Build the product that we have named \lstinline!product! starting from $1$.  
Multiply this by each value $r \in [2 .. N] \equiv [2..N+1)$.
\begin{lstlisting}
def factorial (N):
  product = 1
  if N > 1:
    for r in range(2, N+1):
      product *= r
  return product

print (factorial(6))
\end{lstlisting}
In Line 5 the ``times and becomes'' operator ``\lstinline!product *= r!'' 
requires less typing than ``\lstinline!product = product * r!'', so it is preferred.
\solutionend

\startproblem{Binomial coefficients}
Calculate ``choose $r$ out of $n$ objects'', 
\[\binom{n}{r} = {}_nC^r = \frac{n!}{r!(n-r)!}\]
\problemend
We will discuss binomial coefficients later in Section~\ref{sec:binomialcoefficients}.
\startsolution{Binomial coefficients using factorials}
\label{sol:binomwithfactorials} 
A very slow way of calculating $\binom{n}{r}$ is 
by following the formula slavishly, i.e. divide $n!$ by $r!(n-r)!$.
\begin{lstlisting}
def factorial (n):
  product = 1
  if n > 1:
    for r in range(2, n+1):
      product *= r
  return product

def binom(n, r):
  return factorial(n)/(factorial(r)*factorial(n-r))

print (binom(52,4))
\end{lstlisting}
\solutionend

\startsolution{Binomial coefficients using some mathematics}
Since $n! =( n \times n-1 \times \ldots \times n - r + 1 ) \times (n-r)!$, 
first dividing $n!$ by $(n-r)!$ yields 
\begin{eqnarray*}
n^{\underline{r}} &=& n\times n-1\times \ldots n - r +1 \\
\end{eqnarray*}
The symbol $n^{\underline{r}}$ is called a falling factorial and is 
an abbreviation for $n\times n-1\times \ldots n - r +1$.
What remains, is to calculate $n^{\underline{r}}/r!$.  Noticing that 
the one of the numbers in the product $n \times n-1$ is divisible 
by $2$ and next seeing that one of the factors in $n \times n-1 \times n -2$ 
must be divisible by $3$ leads us to the idea of calculating the binomial 
coefficient as an accumulating product.   First put \lstinline!binom = 1!
and then letting $n$ decrease by $1$ in each iteration, and $r$ starting at $0$ 
increase by $1$ in each iteration accumulate the binomial product as
\begin{lstlisting}
  binom = n
  for r in range(2, R+1):
    n -= 1
    binom = binom * n // r
\end{lstlisting}
This method uses $r-1$ multiplications and divisions and the product 
never gets bigger than the result.  The method in Solution~\ref{sol:binomwithfactorials} 
gets the factorials with close to $2n$ multiplications and one division.
Some simple arithmetic provides a method that works faster for any $n > r$.

\begin{lstlisting}
def binom(n, r):
  binom = n
  for k in range (2, r+1):
    n -= 1
    binom = binom * n // k

  return binom
\end{lstlisting}
\solutionend
Another very interesting solution based solely on additions is presented in 
Solution\ref{sol:binomialadditions}
\startproblem{Counting digits in numbers}
Count the total number digits of each factorial from $1$ to $100000$.
\problemend

\noindent The ultimate Pythonic way to write the factorial function is to use a 
generator.   The generator yields values one by one, discarding the 
previous values.  

\startsolution{$N!$ with a generator}
Our generator for \lstinline!factorials! is a simple variation
of the iterative version of the function.  When control arrives at 
the \lstinline!yield! command the calculation is suspended and the 
environment is saved so that it can be restored on reentry.
\begin{lstlisting}
def factorials (N):
  product = 1
  num = 1 
  while num <= N:
    yield product
    num += 1
    product *= num
     
n = 1000
count = 0
for f in factorials(n):
  count += len(str(f)) 

print (count)
\end{lstlisting}
\solutionend

Many problems lend themselves to recursive solutions.  These recursive 
solutions often lend themselves to be readily turned into iterative 
solutions, e.g. factorials, Fibonacci numbers and sorting.  Let us 
consider those together with their more practical iterative solutions.

\startproblem{Fibonacci numbers}
The sequence of numbers $0,1,1,2,3,5,8,13,21,34, \ldots$ are called 
the Fibonacci numbers.  Starting with $0$ and $1$ next number in the 
sequence is the sum of the previous two.  
\problemend

\startsolution{Fibonacci numbers recursively}
Fibonacci numbers are defined recursively 
as $F_0 = 0$,  $F_1 = 1$,  $F_{n} = F_{n-1} + F_{n-2}$ so the recursive 
code is pretty easy.
\begin{lstlisting}
def F (N):
   if N == 0:
      return 0
   elif N == 1:
      return 1
   else:
      return F(N-1) + F(N-2)

print (F(50))
\end{lstlisting}
\solutionend

\startproblem{Fibonacci numbers}
There is no need to implement Fibonacci numbers recursively and the 
recursive code runs excessively slowly.  Create an iterative solution.
\problemend

\startsolution{Fibonacci numbers iteratively}
If $N$ is zero return a zero; if $N$ is $1$ return a $1$, and otherwise 
use two variables to hold the oldest and most recent values and return 
their sum after updating them.
\begin{lstlisting}
def F (N):
  if N <= 0:
    return 0
  n = 0
  ancient = 0
  old = 1
  while n < N:
    ancient, old = old, ancient + old
    n += 1
  return ancient

from time import time
for n in range (2, 101):
  start = time()
  print("F(%d) = %d, took %.5f s." % (n, F(n), time() - start))
\end{lstlisting}
This code runs in linear time, much faster than the recursive solution 
that runs in exponential time.
\solutionend

\startproblem{A generator for Fibonacci numbers}
Program a generator version of Fibonacci based on the iterative solution.
\problemend

\startsolution{A generator for Fibonacci numbers}
The trick is to figure out where to put the \lstinline!yield! command.
\begin{lstlisting}
def F (N):
  n = 0
  ancient = 0
  old = 1
  while n <= N:
    yield ancient
    ancient, old = old, ancient + old
    n += 1
  return ancient

from time import time
for n in range (2, 101):
  start = time()
  print("F(%d) = %d, took %.5f s." % (n, F(n), time() - start))
\end{lstlisting}
\solutionend

\startproblem{Sorting a list}
 Construct a program to sort a list $L$ of uniform---all the 
elements are numbers or all of them are strings---items from smallest 
to largest.  The sorting algorithm works as follows.  Remove the first 
element of the list and then concatenate the collection of the elements 
smaller than it, the selected element, and the collection of all those 
greater or equal to it.
\problemend

\startsolution{Sorting a list} 
\begin{lstlisting}
def sort(L):
  if not L:
    return L
  it, rest = L[0], L[1:]
  smaller = [item for item in rest if item < it]
  greaterEqual = [item for item in rest if item >= it]
  return sort(smaller) + [it] + sort(greaterEqual)

from random import random

N = 100
L = [int(random()*N) for x in range(N)]
print (sort(L))
\end{lstlisting}
\solutionend
  

\begin{comment}
By using tail recursion that we will discuss later, we will see how some 
of the obstacles can be avoided.
\end{comment}
\noindent Recursive solutions are sometimes easier to conceive.  An example 
whose iterative solution is far from obvious is the towers of Hanoi 
problem.  The recursive solution is easy.

\startproblem{The towers of Hanoi}
Program the towers of Hanoi problem recursively.  Three pins are presented; 
one has a stack of disks piled on top of one another from the smallest 
on top to the largest at the bottom.  The other two pins are empty.  
The three pins are called `\texttt{from}', `\texttt{help}' and `\texttt{to}'. 
The pins are stacked on the \texttt{from} tower.  The problem is to 
move the disks one-by-one such that a larger disk never lies on top 
of a smaller disk from tower to tower until all the disks are in the 
correct order on the \texttt{to} tower.  
\problemend

\startsolution{The towers of Hanoi}
Assume that there are $N$ disks to be manipulated.  The idea of the 
solution is very simple.  How can the bottom disk be moved?  It can 
only be moved if nothing lies on top of it \textit{and} the pin you 
are moving it to is clear.  How can this be brought about? Obviously
$N-1$ pins need to be moved to the \texttt{help} disk.  Once these
disks are out of the way the largest disk---lying on the \texttt{from}
pin---can be moved to the \texttt{to} pin.  What remains to be done?
The $N-1$ disks on the \texttt{help} pin must be moved onto the 
biggest disk now lying on the \texttt{to} pin.  In all the recursive 
procedures above the problem was reduced to doing something for one item, 
and then repeating the procedure for $N-1$ items.  So we have solved 
the towers of Hanoi problem.  The next step is the most difficult.
What parameters will the procedure need?  The value $N$ seems to be 
essential.  Each separate tower needs its own parameter since our 
procedure must know at each step from where, to where a disk must be moved
and which tower remains.\footnote{Perhaps it is possible to can get 
away with naming only the \texttt{from} and \texttt{to} towers in 
the list of parameters.} We will call the procedure \lstinline!hanoi! 
and give it four parameters.
\begin{lstlisting}
def hanoi (N, from, help, to):
  if N == 1:
    print ("move disk from %s to %s" % (from, to))
  else:
    hanoi (N-1, from, to, help) 
    print ("move disk from %s to %s" % (from, to))
    hanoi (N-1, help, from, to)

hanoi (3, "From", "Help", "To")
\end{lstlisting}
\solutionend


Polynomials are interesting for many reasons.  They form the basis 
of our number system.  They may be used to approximate functions 
as accurately as needed.  Polynomials are useful because they are 
easy to differentiate and integrate.  They play an important role 
in algebra.


\startproblem{Evaluating a polynomial}
A univariate polynomial of degree $n$ in $x$ may be written as 
\[P_n(x) = a_0 + a_1x + a_2x^2 + \ldots + a_{n-1}x^{n-1} + a_nx^n.\]
Univariate polynomials are usually referred to simply as polynomials.  
By writing the same terms the other way round
\[P_n(x) = a_nx^n + a_{n-1}x^{n-1} + \ldots+ a_2x^2  + a_1 x + a_0\]
we get a different view of the same polynomial and in this form putting 
the variable $x = 10$ and restricting the coefficients to the decimal 
digits $d_i \in \{0, 1, 2, 3, 4, 5, 6, 7, 8, 9\}$, the polynomial 
\[P_n(10) = d_n10^n + d_{n-1}10^{n-1} + \ldots+ d_210^2  + d_1 10 + d_0\]
represents the decimal number written as 
\[d_n d_{n-1} \ldots d_2 d_1 d_0\]
\problemend


\startsolution{Evaluating a polynomial}
Suppose the coefficients are stored in an array $a$. Write code for 
evaluating the polynomial $P_n(x)$, called by \lstinline!P(n, a, x)!.  
The value of $n$ may be determined by inside the function 
using \lstinline!n = len(a)!, but the meaning of the call is clearer
if the $n$ is passed as an argument. It may be coded as follows.
\begin{lstlisting}
def P(n, a, x):
  pol = a[0]
  xpowern = 1
  for i in range(1, n+1):
    xpowern *= x
    pol = pol + a[i]*xpowern
  return pol

from time import time
a = [1, 4, 6, 4, 1]
n = 4
for x in range(0, 11):
  start = time()
  print("P%d(%f) = %f, took %.5f s." % (n, x, P(n,a,x), time() - start))
\end{lstlisting}
Another way of writing 
the polynomial is to number the coefficients differently as
\[P_n(x) = a_0x^n + a_1x^{n-1} + \ldots + a_{n-1}x^2 + a_{n-1}x + a_n.\]
Rewrite the code corresponding to the array written with the coefficient of the 
$x^n$ first.
\solutionend

\startsolution{Faster evaluation of a polynomial}
\begin{lstlisting}
def Pn(n, a, x):
  pol = 0
  xpowern = 1
  for i in range(n, -1, -1):
    pol = pol + a[i]*xpowern
    xpowern *= x
  return pol

from time import time
a = [1, 4, 6, 4, 1]
n = 4
for x in range(0, 11):
  start = time()
  print("P%d(%f) = %f, took %.5f s." % (n, x, Pn(n,a,x), time() - start))
\end{lstlisting}
\solutionend


These obvious first attempts both use $2n$ multiplications.  Rewrite the code
to use $n$ multiplications.

\startsolution{Horner's rule for polynomial evaluation}
The insight is to start at the back and do only one multiplication 
per iteration.  The idea is called Horner's rule and uses only $n$ 
multiplications for a polynomial of degree $n$.
\begin{lstlisting}
def P(n, a, x):
  pol = a[n]
  for k in range(n-1, -1, -1):
    pol = a[k] +  pol*x
  return pol

from time import time
a = [1, 4, 6, 4, 1]
n = 4
for x in range(0, 11):
  start = time()
  print("P%d(%f) = %f, took %.5f s." % (n, x, P(n,a,x), time() - start))
\end{lstlisting}
\solutionend

\startproblem{Representation of numbers by polynomials}
A number in base ten, e.g. $1234567$ may be represented as the 
array \lstinline!digits = [1, 2, 3, 4, 5, 6, 7]!.  Use the polynomial 
algorithm to find the value of the number.
\problemend

\startsolution{Representation of numbers by polynomials}
\begin{lstlisting}
def P(n, a, x):
  pol = a[n]
  for k in range(n-1, -1, -1):
    pol = a[k] +  pol*x
  return pol

from time import time
digits = [1, 2, 3, 4, 5, 6, 7]
n = 6
for x in range(0, 11):
  start = time()
  digits.reverse()
  print("P%d(%f) = %f, took %.5f s." % (n, x, P(n,digits,x), time() - start))
\end{lstlisting}
\solutionend

\startproblem{Addition of polynomials}
Given two polynomials, $P_n(x)$ and $Q_n(x)$  write a program to add them, 
i.e. write code to produce \[R_n(x) = P_n(x) + Q_n(x).\]
\problemend

\startproblem{Subtraction of polynomials}
Given two polynomials, $P_n(x)$ and $Q_n(x)$  write a program to subtract them,
i.e. write code to produce \[R_n(x) = P_n(x) - Q_n(x).\]
\problemend

\startproblem{Multiplication of polynomials}
Given two polynomials, $P_n(x)$ and $Q_n(x)$  write a programmed to multiply them.
i.e. write code to produce \[R_n(x) = P_n(x) \times Q_n(x).\]
\problemend

\startproblem{Division of polynomials}
Given two polynomials, $P_n(x)$ and $Q_n(x)$  write a programmer to divide them.
i.e. write code to produce \[R_n(x) = P_n(x) / Q_n(x).\]
\problemend

\startproblem{Polynomials representing numbers} 
The polynomial over the variable $b$ with coefficients conveniently 
called digits $d_i \in [0..b) = \{0, 1, ..., b-1\}$, may be used to 
represent any number in base $b$
\[P_n(b) = d_nb^n + d_{n-1}b^{n-1} + \ldots+ d_2b^2  + d_1 b + d_0\]
Write code to convert a number $P_n(b)$ in base $b$ to a number $Q_n(r)$ in base $r$ 
with digits called $s_i \in [0..c) = \{0, 1, \ldots, c-1\}$.
with the same value.  The problem is to convert the polynomial $P$
\[P = d_nb^n + d_{n-1}b^{n-1} + \ldots+ d_2b^2  + d_1 b + d_0\] into the polynomial $Q$
\[Q = s_nr^n + s_{n-1}r^{n-1} + \ldots+ s_2r^2  + s_1 r + s_0,\] 
so that $P = Q$.
\problemend

\startsolution{Polynomials representing numbers}
 This turns out to be easy.  We are given the polynomial 
$P$ and its coefficients and are asked to extract the ``digits'', $s_i$ from it.
Since $P = Q$, we can do arithmetic on $Q$ as the proxy of $P$, so, 
first copy $Q$ into $P$.  Consider $Q$, 
\[Q = s_nr^n + s_{n-1}r^{n-1} + \ldots+ s_2r^2  + s_1 r + s_0,\] 
All the terms of $Q$ excepting $s_0$ are divisible by $r$, so it is 
clear that we can calculate $s_0$ by
\[s_0 = Q\!\!\!\mod r\]
and then subtract---the digit---$s_0$ from $Q$, divide the result with $r$ and replace 
$Q$ with the result
\[Q = \frac{Q - s[0]}{r} = s_nr^{n-1} + s_{n-1}r^{n-2} + \ldots + s_2r  + s_1.\] 
In Python we do this with the two lines
\begin{lstlisting}
s[0] = Q % r
Q //= r
\end{lstlisting}
The steps start repeating.  At this stage, all the terms of $Q$ excepting 
$s_1$ are divisible by $r$, so it is calculated
\[s_1 = Q\!\!\!\mod r\]
and then subtract the digit $s_1$ from $Q$, divide the result with $r$ and replace 
$Q$ with that
\[Q = \frac{Q - s[1]}{r} = s_nr^{n-2} + s_{n-1}r^{n-3} + \ldots + s_3r  + s_2.\] 
In Python we do this with the two lines
\begin{lstlisting}
s[1] = Q % r
Q //= r
\end{lstlisting}
The process continues while $Q \not= 0$.
The code becomes

\begin{lstlisting}
  s = [0]*(n + 1)
  i = 0
  while Q != 0:
    s[i] = Q % r
    Q //= r
    i += 1
\end{lstlisting}
\solutionend

\chapter{Some problems from various sources}

\startproblem{Flatten a binary search tree rightwise}
You are given an unbalanced binary search tree.  Flatten it into a binary
search tree with only right children.
\problemend

\startproblem{Flatten a binary search tree leftwise}
You are given an unbalanced binary search tree.  Flatten it into a binary
search tree with only left children.
\problemend

\startproblem{Balance a binary search tree}
Balance a binary search tree by rightwise flattening and then turn this into
a balanced tree.
\problemend


\startproblem{Find the sum of the numeric keys in a path from the root to
a given node in a binary search tree}
\problemend

\startproblem{Find an item in a sorted list that has been cut once}
Suppose you have a sorted list of length $n$ that has been rotated by $k$ places.
\problemend


\startproblem{Find kth non-repeated char}
\problemend


\startproblem{Find nth largest number in a stream of numbers}
\problemend


\startproblem{Reverse a linked list}
\problemend


\startproblem{Reverse a doubly-linked list}
\problemend


\startproblem{Find Pythagorean triples}
Find the sum of all Pythagorean integer triples $a,b,c$ with $c^2 = a^2 + b^2$ where 
$0< a \leq N$, $0< b \leq N$ and $0< c \leq N$ and $N \leq 1000$.
\problemend


\startproblem{Given the time give the angle in radians between the short
hand and the long hand of an analogue clock}
\problemend


\startproblem{Program Snakes and ladders}
\problemend


\startproblem{Merge three sorted lists}
\problemend


\startproblem{Merge $n$ sorted lists}
\problemend


\startproblem{Find nearest common ancestor of two nodes in a binary search tree}
\problemend


\startproblem{Find common numbers in two sets}
Given a set of $100$ integers and another set of $1000000$ integers.  Find their
intersection.
\problemend


\startproblem{Find the number of days between two dates}
\problemend


\startproblem{Is a given undirected graph a tree or not?}
\problemend


\startproblem{Is a given undirected graph a tree or not?}
\problemend


\startproblem{Find common data from two very large files that do not fit
into memory simultaneously}
\problemend


\startproblem{Print an $n$-ary tree in breadth first order}
\problemend


\startproblem{Checking if two binary search trees are similar or not}
\problemend


\startproblem{Use command line code to get the common data from three files}
\begin{lstlisting}
cat f1 f2 | sort | uniq -d > tmpfile;
cat f3 tmpfile | sort | uniq -d
\end{lstlisting}
\problemend


\startproblem{You have a jar with balls uniquely numbers from 1 to 100. You
have removed 99 of these balls. What is the number of the remaining ball?}
\problemend


\startproblem{Find telephone numbers like 021-959-3004, or 084-455-3885 in all .html files}
The find below finds numbers of the form +27-123-123-1234.
\begin{lstlisting}
find -name *.html | xargs grep "<\ [+]{0,1}\d{1,2}[-]\d{3}[-]\d{3}[-]\d{4}>/"
\end{lstlisting}
\problemend


\startproblem{}
\problemend


\startproblem{}
\problemend
\begin{enumerate}
\item The ancient Babylonians had a fast formula for calculating the square root of
any number.  This algorithm was described by Heron of Alexandria in his book called
Metrica near the dawn of the Christian era about 100 BCE--100 ACE.

Apply Heron's formula to calculate $\sqrt{A}$:\\
``(1) Start with a guess $g$.\\
\mbox{\hspace{5pt}}(2) If it $g^2$ is close enough to $A$, then use $g$ as the square root.\\
\mbox{\hspace{5pt}}(3) Otherwise $g \leftarrow \frac{g + A/g}{2}$.\\
\mbox{\hspace{5pt}}(4) Repeat from step (2).''\\

In books on numerical mathematics Heron's formula to calculate $\sqrt{A}$ is stated as:\\
Initialize $x_0 = 1$, and \\
iterate $x_{n+1} = (x_n + A/x_n)/2$\\
 until $|x_{n+1}^2-A| < \varepsilon$. \\
Note that careful application of this formula in a program does not require subscripts.
\item Find the first repeated element in an array.\\
\item Find the all the repeated elements in an array.\\
\item Find the largest and and second largest elements in a list of integers.\\
\item Swap the largest and smallest elements in a list of integers.\\
\item Reverse the words of a sentence.\\
\item Convert decimal number to Roman number.\\
\item Remove the $n$-th element from a linked list.\\
\item Remove the last element from a linked list.\\
\item Remove the first element from a linked list.\\
\end{enumerate}




\chapter{Conclusion}
\label{section:conclusion}
One of my favourite books on programming Jon Bentley's~\citeyear{bentley1086}
 \textit{Programming Pearls} is stacked with examples.  


\nocite{*}
\bibliographystyle{plainnat}
\bibliography{problems}
\vfill
\pagebreak
\chapter{AppendixA}
\section{The Standard International (SI) symbols for numbers}
{\small
\begin{tabular}{|l|l|l|l|l|l|l|l|}
\hline
Prefix  & \tiny Symbol  & $1000^m$  & $10^n$  & Decimal  & Short scale  & Long scale  & Since\\
\hline
quetta & Q  & $1000^{10}$  & $10^{30}$  & 1\text{ and thirty zeros} & Nontillion  & Sextrillion  & 2022\\
ronnna & R  & $1000^9$  & $10^{27}$  & 1\text{ and twenty seven zeros} & Octillion  & Quintrillion  & 2022\\
yotta  & Y  & $1000^8$  & $10^{24}$  & 1000000000000000000000000  & Septillion  & Quadrillion  & 1991\\
zetta  & Z  & $1000^7$  & $10^{21}$  & 1000000000000000000000  & Sextillion  & Trilliard  & 1991\\
exa  & E  & $1000^6$  & $10^{18}$  & 1000000000000000000  & Quintillion  & Trillion  & 1975\\
peta  & P  & $1000^5$  & $10^{15}$  & 1000000000000000  & Quadrillion  & Billiard  & 1975\\
tera  & T  & $1000^4$  & $10^{12}$  & 1000000000000  & Trillion  & Billion  & 1960\\
giga  & G  & $1000^3$  & $10^9$  & 1000000000  & Billion  & Milliard  & 1960\\
mega  & M  & $1000^2$  & $10^6$  & 1000000  & Million & & 1873\\
kilo  & k  & $1000^1$  & $10^3$  & 1000  & Thousand  & & 1795\\
hecto  & h  & $1000^{2/3}$  & $10^2$  & 100  & Hundred & & 1795\\
deca  & da  & $1000^{1/3}$  & $10^1$  & 10  & Ten & & 1795\\
      &     & $1000^0$  & $10^0$  & $1$  & One  & --- &\\
deci  & d  & $1000^{-1/3}$  & $10^{-1}$  & 0.1  & Tenth & & 1795\\
centi  & c  & $1000^{-2/3}$  & $10^{-2}$  & 0.01  & Hundredth & & 1795\\
milli  & m  & $1000^{-1}$  & $10^{-3}$  & 0.001  & Thousandth & & 1795\\
micro  & μ  & $1000^{-2}$  & $10^{-6}$  & 0.000001  & Millionth & & 1960\\
nano  & n  & $1000^{-3}$  & $10^{-9}$  & 0.000000001  & Billionth  & Milliardth  & 1960\\
pico  & p  & $1000^{-4}$  & $10^{-12}$  & 0.000000000001  & Trillionth  & Billionth  & 1960\\
femto  & f  & $1000^{-5}$  & $10^{-15}$  & 0.000000000000001  & Quadrillionth  & Billiardth  & 1964\\
atto  & a  & $1000^{-6}$  & $10^{-18}$  & 0.000000000000000001  & Quintillionth  & Trillionth  & 1964\\
zepto  & z  & $1000^{-7}$  & $10^{-21}$  & 0.000000000000000000001  & Sextillionth  & Trilliardth  & 1991\\
yocto  & y  & $1000^{-8}$  & $10^{-24}$  & 0.000000000000000000000001  & Septillionth  & Quadrillionth  & 1991\\
ronto  & y  & $1000^{-9}$  & $10^{-27}$  & 0.\text{ twenty six zeros }1  & Octtillionth  & Quintillionth  & 1991\\
quecto & y  & $1000^{-10}$  & $10^{-30}$  & 0.\text{ twenty nine zeros }1  & Nonillionth  & Sextillionth  & 1991\\
\hline
\end{tabular}}
\tableofcontents 
  % Not necessary for a small document

\end{document}



% compare typesetting 
\[WCETT(p) = (1 - \beta)\sum_{\mbox{link}\in p}ETT_i + \beta \max_{1\leq j\leq k}X_j \]
\[\mbox{\textit{WCETT}}(p) = (1 - \beta)\sum_{\mbox{link}\in p}ETT_i + \beta \max_{1\leq j\leq k}X_j \]
\[\stackrel{\mbox{argmin}}{b}\bigg(\sum_{j=1}^b W_j>T\bigg)\]




